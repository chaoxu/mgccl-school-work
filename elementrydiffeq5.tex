\documentclass[letter]{article}
\usepackage{amsmath}
\usepackage{amssymb}
\usepackage{amsfonts}
\usepackage{amsthm}
\usepackage{latexsym}
\usepackage{mathrsfs}
\usepackage{eufrak}
\usepackage[pdftex]{graphicx}
\usepackage{color}

\setlength{\topmargin}{-0.5in} \setlength{\textwidth}{6.5in}
\setlength{\oddsidemargin}{0.0in} \setlength{\textheight}{9.1in}

\newlength{\pagewidth}
\setlength{\pagewidth}{6.5in} \pagestyle{empty}

\newcommand{\R}{\mathbb{R}}
\newcommand{\N}{\mathbb{N}}
\newcommand{\B}{\mathfrak{B}}
\newcommand{\Z}{\mathbb{Z}}
\newcommand{\im}{\operatorname{im}}
\newcommand{\rref}{\operatorname{rref}}
\newcommand{\rank}{\operatorname{rank}}
%\newcommand{\ker}{\operatorname{ker}}
\newtheorem{theorem}{Theorem}[section]
\newtheorem{lemma}[theorem]{Lemma}
\newtheorem{proposition}[theorem]{Proposition}
\newtheorem{corollary}[theorem]{Corollary}

\newenvironment{definition}[1][Definition]{\begin{trivlist}
\item[\hskip \labelsep {\bfseries #1}]}{\end{trivlist}}
\newenvironment{example}[1][Example]{\begin{trivlist}
\item[\hskip \labelsep {\bfseries #1}]}{\end{trivlist}}
\newenvironment{remark}[1][Remark]{\begin{trivlist}
\item[\hskip \labelsep {\bfseries #1}]}{\end{trivlist}}

\title{Elementary Differential Equations(Boyce) Chapter 5(Series solutions of Second Order Linear Equations) Notes}
\date{}

\begin{document}
\maketitle
\vspace{-.5in}

\section{Review of Power Series}
\begin{definition}
A power series $\sum_{n=0}^{\infty} a_n(x-x_0)^n$ is said to \textbf{converge} at a point $x$ if
\[
\lim_{m\to \infty} \sum_{n=0}^{m} a_n(x-x_0)^n
\]
exists. The series converges for $x=x_0$; it may converge for all $x$ or some $x$.
\end{definition}

\begin{definition}
A power series $\sum_{n=0}^{\infty} a_n(x-x_0)^n$ is said to \textbf{converge absolutely} at a point $x$ if
\[
\sum_{n=0}^{\infty} |a_n(x-x_0)^n|
\]
exists. If the series converges absolutely, then the series converge.
\end{definition}

\begin{theorem}
The most useful test for the absolute convergence of a power series is the ratio test. If $a_n\neq 0$, and if for a fixed value of $x$
\[
\lim_{n\to \infty} |\frac{a_{n+1}(x-x_0)^{n+1}}{a_n(x-x_0)^n}| = |x-x_0| \lim_{n\to \infty} |\frac{a_{n+1}}{a_n} = l|
\]
then the power series converges absolutely at that value of $x$ if $l < 1$, and diverges if $l>1$. If $l=1$ the test is inconclusive.
\end{theorem}

\begin{theorem}
If the power series  $\sum_{n=0}^{\infty} a_n(x-x_0)^n$ converges at $x = x_1$, it converges absolutely for $|x-x_0| < |x_1 - x_0|$; and if it diverges at $x=x_1$, it diverges for $|x-x_0| > |x_1-x_0$.
\end{theorem}

\begin{definition}
There is a nonnegative number $\rho$, called the \textbf{radius of convergence}, such that  $\sum_{n=0}^{\infty} a_n(x-x_0)^n$ converges absolutely for $|x-x_0|<\rho$ and diverges for $|x-x_0|>\rho$. For a series that converges nowhere except at $x_0$, we define $\rho$ to be zero; for a series that converges for all $x$, we say that $\rho$ is infinite. If $\rho>0$, then the interval $|x-x_0<\rho|$ is called the \textbf{interval of convergence}. The series may either converge or diverge when $|x-x_0| = \rho$.
\end{definition}

If  $\sum_{n=0}^{\infty} a_n(x-x_0)^n$ and  $\sum_{n=0}^{\infty} b_n(x-x_0)^n$ converge to $f(x)$ and $g(x)$, respectively, for $|x-x_0|<\rho$, $\rho>0$, then the following are true for $|x-x_0|<\rho$.
\[
f(x) \pm g(x) = \sum_{n=0}^{\infty} (a_n\pm b_n)(x-x_0)^n
\]
\[
f(x)g(x) = \sum_{n=0}^{\infty} (c_n)(x-x_0)^n
\]
where $c_n = \sum_{i=0}^n a_i b^{n-i}$
\[
\frac{f(x)}{g(x)} = \sum_{n=0}^{\infty} (d_n)(x-x_0)^n
\]
which can be done by using below.
\[
\sum_{n=0}^{\infty} (a_n)(x-x_0)^n = \sum_{n=0}^{\infty} (d_n)(x-x_0)^n\sum_{n=0}^{\infty} (b_n)(x-x_0)^n
\]
In case of division, the radius of convergence of the resulting power series may be less than $\rho$

\begin{theorem}
$f$ is continuous and has derivatives of all orders of $|x-x_0|<\rho$. Further, $f',f'',\ldots$ can be computed by differentiating the series term wise; that is,
\[f'(x) = \sum_{n=1}^\infty n a_n (x-x_0)^{n-1}\]
\[f''(x) = \sum_{n=2}^\infty n(n-1) a_n (x-x_0)^{n-2}\]
and so forth, and each of the series converges absolutely for $|x-x_0|<\rho$.
\end{theorem}

\begin{definition}
\[ a_n = \frac{f^{(n)}(x_0)}{n!} \]
The series is called the Taylor series for $f$ about $x=x_0$.
\end{definition}

\begin{theorem}
If $\sum_{n=0}^\infty a_n(x-x_0)^n = \sum_{n=0}^\infty b_n(x-x_0)^n$ for each $x$, then $a_n = b_n$. If $\sum_{n=0}^\infty a_n(x-x_0)^n = 0$ for each $x$, then $a_n = 0$.
\end{theorem}

\begin{definition}
\[
f(x) = \sum_{n=0}^\infty \frac{f^{(n)}(x_0)}{n!}(x-x_0)^n
\]
with a radius of convergence $\rho > 0$ is said to be analytic at $x=x_0$.
\end{definition}


\end{document}
\theend
