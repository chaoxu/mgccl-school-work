\documentclass[letter]{article}
\usepackage{enumerate}
\usepackage{amsmath}
\usepackage{amssymb}
\usepackage{amsfonts}
\usepackage{amsthm}
\usepackage{latexsym}
\usepackage{mathrsfs}
\usepackage{eufrak}
\usepackage[pdftex]{graphicx}
\usepackage{color}
\usepackage{mathrsfs}
\usepackage[utf8]{inputenc}
\usepackage{tikz}
\usetikzlibrary{snakes,arrows,shapes}
\setlength{\topmargin}{-0.5in} \setlength{\textwidth}{6.5in}
\setlength{\oddsidemargin}{0.0in} \setlength{\textheight}{9.1in}
\usepackage[inline]{asymptote}
\usepackage{dot2texi}

\newlength{\pagewidth}
\setlength{\pagewidth}{6.5in} \pagestyle{empty}

\newcommand{\R}{\mathbb{R}}
\newcommand{\N}{\mathbb{N}}
\newcommand{\B}{\mathfrak{B}}
\newcommand{\Z}{\mathbb{Z}}
\newtheorem{theorem}{Theorem}[section]
\newtheorem{lemma}[theorem]{Lemma}
\newtheorem{proposition}[theorem]{Proposition}
\newtheorem{corollary}[theorem]{Corollary}

\newenvironment{definition}[1][Definition]{\begin{trivlist}
\item[\hskip \labelsep {\bfseries #1}]}{\end{trivlist}}
\newenvironment{example}[1][Example]{\begin{trivlist}
\item[\hskip \labelsep {\bfseries #1}]}{\end{trivlist}}
\newenvironment{remark}[1][Remark]{\begin{trivlist}
\item[\hskip \labelsep {\bfseries #1}]}{\end{trivlist}}

\title{AMS 311 Spring 2010 HW \#2}
\date{}
\author{Chao Xu}
\begin{document}
\maketitle
\vspace{-.5in}
\section*{Problems}
\subsection*{4}
\paragraph*{(a)}
The sample space consist of all finite strings that contain one $1$ in the end and a infinite string of only $0$'s. Where $0$ is tail and $1$ is head.
\paragraph*{(b)}
\begin{enumerate}[(i)]
 \item The strings with the amount of zeros that are divisible by 3.
 \item The strings with the amount of zeros = 1 mod 3.
 \item The strings with the amount of zeros = 2 mod 3.
\end{enumerate}

\subsection*{13}
\paragraph*{(a)}
$100000 (P(I\cup II\cup III \cap)-P(I\cap III) -P(I\cap II)- P(II\cap III) + P(I\cap II \cap III)) =100000((0.1 + 0.3 + 0.05 - 0.08 - 0.02 - 0.04 + 2*0.01) - 0.08 - 0.02 - 
  0.04 + 0.01) = 20000$
\paragraph*{(b)}
$100000 P((I\cap II)\cup (I\cap III) \cup (III\cap II)) = 100000(0.08+0.02+0.04-2*0.01) = 12000$
\paragraph*{(c)}
$100000 P((I\cup III) \cap II) = 100000 P((I\cap II) \cup (III\cap II)) = 100000(0.08+0.04-0.01) = 11000$
\paragraph*{(d)}
$100000(1-P(I\cup II\cup III)) = 100000(1-(0.1+0.3+0.05-0.08-0.02-0.04+0.01)) = 68000$
\paragraph*{(e)}
$100000 (P((I\cap II) \cup (III\cap II)) - P(I \cap II \cap III)) = 100000(0.08+0.04-0.01-0.01) = 10000$

\subsection*{15}
\paragraph*{(a)}
$\frac{4{13 \choose 5}}{{52 \choose 5}}\approx 0.00198079$
\paragraph*{(b)}
Ways to get a pair = ${13 \choose 1}{4\choose 2}$. Ways to get 3 different card after getting a pair = $\frac{48\cdot 44\cdot 40}{3!}$.\\
Probability = $\frac{{13 \choose 1}{4\choose 2}48\cdot 44\cdot 40}{{3!}{52\choose 5}}\approx 0.422569$.
\paragraph*{(c)}
Ways to get 2 pairs = ${13 \choose 1}{4\choose 2}{12 \choose 1}{4\choose 2}/2!$. Ways to get 1 card after 2 pairs = $44$\\
Probability = $\frac{44{13 \choose 1}{4\choose 2}{12 \choose 1}{4\choose 2}}{2{52 \choose 5}}\approx 0.047539$
\paragraph*{(d)}
Similar to above.
$\frac{{13 \choose 1}{4\choose 3}48\cdot 44}{2{52 \choose 5}}\approx 0.0211285$
\paragraph*{(e)}
$\frac{{13 \choose 1}{4\choose 4}48}{{52 \choose 5}}\approx 0.000240096$

\subsection*{26}
Table of $P(R_i)$.
\begin{tabular}{|c|c|}
\hline
$i$&$P(R_i)$\\
\hline
$2$&$\frac{1}{36}$\\
\hline
$3$&$\frac{2}{36}$\\
\hline
$4$&$\frac{3}{36}$\\
\hline
$5$&$\frac{4}{36}$\\
\hline
$6$&$\frac{5}{36}$\\
\hline
$7$&$\frac{6}{36}$\\
\hline
$8$&$\frac{5}{36}$\\
\hline
$9$&$\frac{4}{36}$\\
\hline
$10$&$\frac{3}{36}$\\
\hline
$11$&$\frac{2}{36}$\\
\hline
$12$&$\frac{1}{36}$\\
\hline
\end{tabular}

Let $P(E_{i,n})$ = initial sum is $i$, and player win at $n$-th throw. $P(E_i)$ = probabilty of winning the when the initial sum is $i$. $P(R_i)$ is the probability to row a $i$.
Then probabilty of winning is\\
\[\sum_{i=2}^{12} P(E_i) = P(R_7)+P(R_{11}) + \sum_{i\in\{4,5,6,8,9,10\}} P(E_i)\]
\[= P(R_7)+P(R_{11}) + \sum_{i\in\{4,5,6,8,9,10\}} \sum_{n=1}^\infty P(E_i,n)\]
\[= P(R_7)+P(R_{11}) + \sum_{i\in\{4,5,6,8,9,10\}} P(R_i)\sum_{n=2}^\infty P(R_i)(1-P(R_7)-P(R_i))^{n-2}\]
\[= P(R_7)+P(R_{11}) + 2\sum_{i=4}^6 P(R_i)^2 \sum_{n=2}^\infty (1-P(R_7)-P(R_i))^{n-1}\]
\[=\frac{6}{36}+\frac{2}{36}+2\sum_{i=4}^6 P(R_i)^2\sum_{n=2}^\infty (\frac{30}{36}-P(R_i))^{n-2}\]
\[=\frac{8}{36}+2\sum_{i=3}^5 (\frac{i}{36})^2 \sum_{n=0}^\infty (\frac{30-i}{36})^{n}\]
\[=\frac{8}{36}+2\sum_{i=3}^5 (\frac{i}{36})^2\frac{36}{6+i}\]
\[=\frac{8}{36}+2(\frac{3}{36}\frac{3}{36}\frac{36}{9} + \frac{4}{36}\frac{4}{36}\frac{36}{10} + \frac{5}{36}\frac{5}{36}\frac{36}{11})\]
\[=\frac{244}{495}\approx 0.492929\]
\subsection*{33}
There are ${20\choose 4}$ sets of 4 elks.\\
There are ${5 \choose 2}$ sets of 2 elks from the tagged group. ${15 \choose 2}$ sets of 2 elks from the non-tagged group.\\
Probability is $\frac{{5 \choose 2}{15\choose 2}}{{20\choose 4}}\approx 0.216718$\\
Assumption: The probability of any elk been captured is the same.
\subsection*{35}
\paragraph*{(a)}
$\frac{{12\choose 3}{16\choose 2}{18\choose 2}}{{46\choose 7}}\approx 0.0754643$
\paragraph*{(b)}
$1 - \frac{{34\choose 7}+{34\choose 6}{12\choose 1}}{{46\choose 7}} \approx 0.597971$
\paragraph*{(c)}
$\frac{{12\choose 7}+{16\choose 7}+{18\choose 7}}{{46 \choose 7}}\approx 0.000823097$
\paragraph*{(d)}
$
\frac{{12\choose 3}{34\choose 4}+{16\choose 3}{30\choose 4}-{12\choose 3}{16\choose 3}{18\choose 1}}{{46\choose 7}}\approx 0.43591
$
\subsection*{51}
There are ${n \choose m}$ ways to chose $m$ balls, and the rest $n-m$ have to fall in $N-1$ compartments, which is $(N-1)^{n-m}$. Thus it become ${n \choose m}(N-1)^{n-m}$ for $m$ to fall into the first compartment, and rest fall elsewhere. Then the probability is $\frac{{n\choose m}(N-1)^{n-m}}{N^n}$.

\subsection*{54}
void in 1 suit = ${4 \choose 1}{39 \choose 13}$\\
void in 2 suit = ${4 \choose 2}{26 \choose 13}$\\
void in 3 suit = ${4 \choose 3}{13 \choose 13}$\\
void in at least 1 suit = void in 1 suit - void in 2 suit + void in 3 suit.\\
probability of void in at least 1 suit =$\frac{{4 \choose 1}{39 \choose 13}-{4 \choose 2}{26 \choose 13}+{4 \choose 3}{13 \choose 13}}{{52\choose 13}}\approx 0.0510655$

\section*{Theoretical Exercises}

\subsection*{17}
Let $P(n)$ be the predicate ``$A_n = (n-1)(A_{n-1}+A_{n-2})$''.\\
\emph{base case}: $P(3)$ is true, can be proved by list all possible cases.\\
\emph{inductive step}: Suppose all $P(k)$ is true where $k<n$.Then prove $P(n)$ is true.\\
The first person will pick a hat that's not his, he have $(n-1)$ choices. We have:
$A_n = (n-1)f(n)$.
We have a extra person, and there is a extra hat, if he chose the extra hat, there is only 1 choice, and then it is reduced to ways for $n-2$ people to chose $n-2$ hats = $A_{n-2}$\\
A group of $n-1$ people, where $n-2$ of them have their own hat, and one extra person and extra hat but doesn't chose that hat. Non of them chose the same hat be $g(n-1)$.\\
$A_n = (n-1)(A_{n-2} + g(n-1))$\\
If the extra person didn't chose the extra hat, he have $(n-2)$ ways to chose. Then we have 1 extra person with 1 extra hat.\\
$A_n = (n-1)(A_{n-2} + (n-2)(A_{n-3}+g(n-2))$.\\
We have\\
$g(n-1) = (n-2)(A_{n-3}+g(n-2))$\\
Then one just have to find a suitable $g(n)$ such the above relation is true for all $g(k)$, $k<n$.\\
If $g(k) = A_{k}$,then the relation $g(k-1) = (k-2)(A_{k-3}+g(k-2))$ is true for all $g(k)$, $k<n$\\
Substute $A_{n-1}$ for $g(n-1)$. $A_n = (n-1)(A_{n-2} + A_{n-1})$\\
Thus $P(n)$ is true for all $n$.


\subsection*{20}
Let the points be $a_1,a_2,\ldots$, and the probability of chosing poing $a_i$ be $P(a_i)$. If they are all equal, then there exist $p$, such that $\forall i(p = P(a_i))$
\[\sum_{i=1}^\infty P(a_i) = \sum_{i=1}^\infty p = 1\]
Such real number $p$ can't exist. For any real number $p>0$, there exist a $n$ such that $np>1$ due to the Archimedean property of real numbers. Thus \[\sum_{i=1}^n p > 1\] for some $n$. for $p\leq 0$, \[\sum_{i=1}^\infty p \leq 0\]
It's possible for all points to have a positive probability.\\
For example, $P(a_i) = \frac{1}{2^i}$
\end{document}
\theend
