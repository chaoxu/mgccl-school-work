\documentclass[letter]{article}
\usepackage{amsmath}
\usepackage{amssymb}
\usepackage{amsfonts}
\usepackage{amsthm}
\usepackage{latexsym}
\usepackage{mathrsfs}
\usepackage{eufrak}
\usepackage[pdftex]{graphicx}
\usepackage{color}

\setlength{\topmargin}{-0.5in} \setlength{\textwidth}{6.5in}
\setlength{\oddsidemargin}{0.0in} \setlength{\textheight}{9.1in}

\newlength{\pagewidth}
\setlength{\pagewidth}{6.5in} \pagestyle{empty}

\newcommand{\R}{\mathbb{R}}
\newcommand{\N}{\mathbb{N}}
\newcommand{\B}{\mathfrak{B}}
\newcommand{\U}{\mathfrak{U}}
\newcommand{\Z}{\mathbb{Z}}
\newcommand{\im}{\operatorname{im}}
\newcommand{\image}{\operatorname{image}}
\newcommand{\rref}{\operatorname{rref}}
\newcommand{\rank}{\operatorname{rank}}
\newcommand{\nullity}{\operatorname{nullity}}
%\newcommand{\ker}{\operatorname{ker}}
\newtheorem{theorem}{Theorem}[section]
\newtheorem{lemma}[theorem]{Lemma}
\newtheorem{proposition}[theorem]{Proposition}
\newtheorem{corollary}[theorem]{Corollary}

\newenvironment{definition}[1][Definition]{\begin{trivlist}
\item[\hskip \labelsep {\bfseries #1}]}{\end{trivlist}}
\newenvironment{example}[1][Example]{\begin{trivlist}
\item[\hskip \labelsep {\bfseries #1}]}{\end{trivlist}}
\newenvironment{remark}[1][Remark]{\begin{trivlist}
\item[\hskip \labelsep {\bfseries #1}]}{\end{trivlist}}

\title{Linear Algebra(Bretscher) Chapter 6 Notes}
\date{}

\begin{document}
\maketitle
\vspace{-.5in}

\section{Introduction to Determinants}
\subsection{The Determinant of a $3\times 3$ Matrix}
\begin{definition}
If $A = \begin{bmatrix}
\vec u & \vec v & \vec w
\end{bmatrix}$, then
\[
\det A = \vec u \cdot (\vec v \times \vec w)
\]
\end{definition}

\begin{theorem}
\[\det A =  a_{11}a_{22}a_{33} + a_{12}a_{23}a_{31} + a_{13}a_{22}a_{31} - a_{13}a_{22}a_{31} - a_{11}a_{23}a_{32} - a_{12}a_{21}a_{33}\]
\end{theorem}

\subsection{Linearity properties of the determinant}
The determinant are linear if all but one column or row vector is constant.

\subsection{The Determinant of an $n \times n$ Matrix}
\begin{definition}
A \textbf{pattern} in an $n \times n$ matrix $A$ is a way to choose $n$ entries of the matrix so that there is one chosen entry in each row and in each column of $A$.
\end{definition}

\begin{definition}
Two entries in a pattern are said to be \textbf{inverted} if one of them is located to the right and above the other in the matrix.
\end{definition}

\begin{definition}
The \textbf{signature} of a pattern $P$ is defined as $\operatorname{sgn} P = (-1)^{(\text{number of inversions in $P$})}$
\end{definition}

\begin{definition}
\[ \det A = \sum (\operatorname{sgn} P) (\operatorname{prod} P) \]
where $\operatorname{prod} P$ is the product of the entires in $P$.
\end{definition}

\section{Properties of the Determinant}
\subsection{The determinant of the Transpose}
\begin{theorem}[The determinant of the Transpose]
If $A$ is a square matrix, then
\[
\det(A^T) = \det A
\]
\end{theorem}

\subsection{Linearity properties of the Determinant}
\begin{theorem}[Linearity of the determinant in the rows and columns]
Consider fixed row vectors $\vec v_1, \ldots, \vec v_{i-1}, \vec v_{i+1},\ldots,\vec v_n$ with $n$ components. Then the function

\[T(\vec x) = \det 
\begin{bmatrix}
\vec v_1\\
\ldots\\
\vec v_{i-1}\\
\vec x\\
\vec v_{i+1}\\
\ldots\\
\vec v_n
\end{bmatrix}
\]
from $\R^{1\times n}$ to $\R$ is a linear transformation. This property is referred to as \textbf{linearity of the determinant in the $i$th row}. Likewise, the determinant is \textbf{linear in all the columns}.
\end{theorem}

\subsection{Determinants and Gauss-Jordan Elimination}
\begin{theorem}[Elementary row operations and determinants]
\begin{enumerate}
\item If $B$ is obtained from $A$ by dividing a row of $A$ by scalar $k$, then
\[
\det B = (1/k) \det A
\]
\item If $B$ is obtained from $A$ by a row swap, then
\[
\det B = - \det A
\]
we say that the determinant is \textbf{alternating} on the rows.
\item If $B$ is obtained from $A$ by adding a multiple of a row of $A$ to another row, the determinant doesn't change.
\end{enumerate}
\end{theorem}

\begin{theorem}[Determinant of a product of matrices]
If $A$ and $B$ are $n\times n$ matrices, then
\[
\det(AB) = (\det A)(\det B)
\]
\end{theorem}

\begin{theorem}[Determinants of similar matrices]
If matrix $A$ is similar to $B$, then $\det A = \det B$.
\end{theorem}

\begin{theorem}[Determinant of a inverse]
If $A$ is an invertible matrix, then
\[
\det(A^{-1}) = \frac{1}{\det A} = (\det A)^{-1}
\]
\end{theorem}
\end{document}
\theend