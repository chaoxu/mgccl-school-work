\documentclass[letter]{article}
\usepackage{amsmath}
\usepackage{amssymb}
\usepackage{amsfonts}
\usepackage{latexsym}
\usepackage{amsthm}
\usepackage[pdftex]{graphicx}
\usepackage{color}
\usepackage{xspace}
\usepackage[x11names, rgb]{xcolor}
\usepackage[utf8]{inputenc}
\usepackage{tikz}
\usetikzlibrary{snakes,arrows,shapes}

\setlength{\topmargin}{-0.5in} \setlength{\textwidth}{6.5in}
\setlength{\oddsidemargin}{0.0in} \setlength{\textheight}{9.1in}

\newlength{\pagewidth}
\setlength{\pagewidth}{6.5in} \pagestyle{empty}

\newcommand{\R}{\mathbb{R}}
\newcommand{\N}{\mathbb{N}}
\newcommand{\Z}{\mathbb{Z}}
\newtheorem{lma}{Lemma}
\author{Chao Xu}
\title{AMS 311 Spring 2010: Homework \#1}
\date{}
\begin{document}
\maketitle
\vspace{-.5in}
\section*{Problems}
\subsection*{3}
$20!$
\subsection*{10}
\paragraph*{(a)} $8!=40320$
\paragraph*{(b)} $2\times 7!= 10080$
\paragraph*{(c)} $2\times 4!=48$
\paragraph*{(d)} $4!\times 5!=2880$
\paragraph*{(e)} $4!\times 2^4 =384$
\subsection*{18}
${5 \choose 2} {6 \choose 2} {4 \choose 3}=600$
\subsection*{25}
If 4 players are distinguishable, then ${52 \choose 13,13,13,13} =
\frac{52!}{(13!)^4}$. Else it's ${52 \choose 13,13,13,13} / 4! =
\frac{52!}{(13!)^4 4!}$
\section*{Theoretical Exercises}
\subsection*{2}
\[
\sum_{i=1}^m n_i
\]
\subsection*{8}
Let $N,M$ be disjoint sets. $|N| = n$, $|M| = m$. $R\subset (N \cup M)$ and
$|R|=r$. Then the amount of $R$ is ${n+m \choose r}$.\\
We can find $R_1 \subseteq R-M$ and $R_2 \subseteq R-N$ where $R_1 \cup R_2 =
R$ and $R_1 \cap R_2 = \emptyset$. Let $|R_1|=i$, then $|R_2|=r-i$. $R_1$ is
made from chose $i$ elements from $N$, and $R_2$ is made from chose $r-i$
elements from $M$. Summing up ${n \choose i}{m \choose r-i}$ over all $i$ result
the amount of possible $R$. Which is $\sum_{i=0}^r {n \choose i}{m \choose
r-i}$.
\subsection*{13}
\begin{align}
(1-1)^n&=\sum_{i=0}^n {n \choose i} 1^{n-i}(-1)^i\\
&=\sum_{i=0}^n {n \choose i}(-1)^i\\
&=0
\end{align}

\section*{Self-test problems and exercises}
\subsection*{4}
There are ${10 \choose 7}=120$ ways to answer 7 out of 10 questions.\\
There are ${5 \choose 3}{5 \choose 4}+{5 \choose 4}{5 \choose 3}+{5 \choose 5}{5
\choose 2}=110$ ways if she have to answer at least 3 out of the first 5
questions.

\subsection*{9}
\paragraph*{(a)}
${3n \choose 3}$
\paragraph*{(b)}
${n \choose 3}{3 \choose 1}$
\paragraph*{(c)}
${n \choose 2}{n \choose 1}{3 \choose 2}$
\paragraph*{(d)}
${n\choose 1}{n \choose 1}{n \choose 1}{3 \choose 3}$
\paragraph*{(e)}
${3n \choose 3} = {n \choose 3}{3 \choose 1}+{n \choose 2}{n \choose 1}{3
\choose 2}+{n\choose 1}{n \choose 1}{n \choose 1}{3 \choose 3}$

\subsection*{18}
$3 \times 1 \times 1+ 3\times 1 \times 2 + 5 \times 2 \times 1 + 7\times 2
\times 2 + 6\times 2 \times 3 = 83$

\section*{Extra Credit}
\subsection*{32*}
$a_1+a_2+a_3+a_4+a_5+a_6 = 8$, where $a_i$ is the amount of people left at
$i$-th floor. Thus it
become ${13 \choose 5} = 1287$.\\
For the second case, we try to find $a_1+a_2+a_3+a_4+a_5+a_6 = 3$ and
$b_1+b_2+b_3+b_4+b_5+b_6 = 5$. Which are ${8 \choose 5}$ and ${10 \choose 5}$
respectively. Since both are independent, the amount of ways are ${8 \choose
5}{10 \choose 5} = 14112$
\end{document}
\theend
