\documentclass[letter]{article}
\usepackage{amsmath}
\usepackage{amssymb}
\usepackage{amsfonts}
\usepackage{amsthm}
\usepackage{latexsym}
\usepackage[pdftex]{graphicx}
\usepackage{color}

\setlength{\topmargin}{-0.5in} \setlength{\textwidth}{6.5in}
\setlength{\oddsidemargin}{0.0in} \setlength{\textheight}{9.1in}

\newlength{\pagewidth}
\setlength{\pagewidth}{6.5in} \pagestyle{empty}

\newcommand{\R}{\mathbb{R}}
\newcommand{\N}{\mathbb{N}}
\newcommand{\Z}{\mathbb{Z}}

\newtheorem{theorem}{Theorem}[section]
\newtheorem{lemma}[theorem]{Lemma}
\newtheorem{proposition}[theorem]{Proposition}
\newtheorem{corollary}[theorem]{Corollary}

\newenvironment{definition}[1][Definition]{\begin{trivlist}
\item[\hskip \labelsep {\bfseries #1}]}{\end{trivlist}}
\newenvironment{example}[1][Example]{\begin{trivlist}
\item[\hskip \labelsep {\bfseries #1}]}{\end{trivlist}}
\newenvironment{remark}[1][Remark]{\begin{trivlist}
\item[\hskip \labelsep {\bfseries #1}]}{\end{trivlist}}

\title{Zorich Chapter 2: The Real Numbers Notes}
\date{}

\begin{document}
\maketitle
\vspace{-.5in}
\section{The axiom System and some General Properties of the Set of Real Numbers}
\subsection{Definition of the Set of Real Numbers}
\newtheorem{axi}{Axiom}
\begin{definition}
Set $\R$ is called the set of \textbf{real numbers} if:\\
1. Axioms for addition\\
2. Axioms for multiplication\\
(1,2). Multiplication is distributive with respect to addition\\
3. Order Axioms\\
(1,3). $(x\leq y) \implies (x+z\leq y+z)$\\ 
(2,3). $(0\leq x) \wedge (0\leq y) \implies (0\leq xy)$\\
4. The axiom of completeness (continuity)\\
All satisfied.
\end{definition}

\begin{axi}[Axiom for addition]
An operation
\[ +: \R \times \R \to \R \]
is defined, such that.
Each ordered pair $(x,y)$ map to $x=y \in \R$, The condition of the following is met.
\begin{enumerate}
\item There exist a identity 0, such that
\[
x+0 = x
\]
for all $x \in \R$
\item There exist $-x\in \R$ for every $x$ such that
\[ 
x+(-x) = 0
\]
\item + is associate 
\[
x+(y+z) = (x+y)+z
\]
\item + is commutative
\[
x+y = y+x
\]
\end{enumerate}
\end{axi}

\begin{definition}
A \textbf{group} is a set satisfies Axioms for addition except commutativity.
\end{definition}

\begin{definition}
A \textbf{commutative group} is a set satisfies Axioms for addition.
\end{definition}

\begin{axi}[Axioms for multiplication]
Operation 
\[
\cdot: \R \times \R \to \R
\]
 is defined, such that
\begin{enumerate}
\item There exist a identity $1\in \R \setminus 0$, such that
\[
x\cdot 1 = x
\]
\item There exist $x^{-1} \in \R$ for every $x\in \R \setminus 0$ such that
\[
x\cdot x^{-1} = 1
\]
\item $\cdot$ is associative
\item $\cdot$ is commutative
\end{enumerate}
\end{axi}

Axioms for multiplication also defines a group structure on $\R \setminus 0$. Later $x\cdot y$ is also written as $xy$ if there is no ambiguity.

\begin{definition}
A \textbf{field} is a set satisfy Axioms for addition, multiplication and multiplication is distributive with respect to addition.
\end{definition}

\begin{axi}[Order Axioms]
A relation $\leq$ for elements $x,y\in \R$, here is the conditions must hold.
\begin{enumerate}
\item $\forall x \in \R (x\leq x)$
\item $(x\leq y) \wedge (y\leq x) \implies (x=y)$
\item $(x\leq y) \wedge (y\leq z) \implies (x\leq z)$
\item $\forall x \in \R \wedge \forall y \in \R (x\leq y) \vee (y \leq x)$
\end{enumerate}
\end{axi}

\begin{definition}
A set satisfying Order Axioms except the last one is said to be \textbf{partially ordered}.
\end{definition}

\begin{definition}
A set satisfying Order Axioms is said to be \textbf{linearly ordered}.
\end{definition}

\begin{axi}[Axiom of completeness(continuity)]
If $X$ and $Y$ are nonempty subset of $\R$ having the property $x \leq y$ for every $x \in X$ and every $y\in Y$. Then there exist $c\in R$ such that $x\leq c\leq y$ for all $x \in X$ and $y \in Y$.
\end{axi}

page 38 have a nice discussion on axioms.

\begin{definition}
If a set satisfying an axiom system, then the set is \textbf{consistent}.
\end{definition}
\begin{definition}
If an axiom system determine an object uniquely, then it is \textbf{categorical}.
\end{definition}
\begin{definition}
If set $A$ and set $B$ satisfying the axiom system, and there is a bijection from $A$ to $B$ and preserve the axiom system. Then the mapping is called an \textbf{isomorphism} and the $A$ and $B$ are \textbf{isomorphic}
\end{definition}
\begin{theorem}
Real numbers are consistent from ZFC.
\end{theorem}
\begin{theorem}
Real numbers is categorical.
\end{theorem}

\subsection{Some General Algebraic Properties of Real Numbers}
\begin{enumerate}
\item Consequences of the Addition Axioms.
	\begin{enumerate}
		\item There is only one additive identity in the set of real numbers.
		\item There is only one negative for each number.
		\item $a+x=b$ has a unique solution, $x = b+(-a)$
	\end{enumerate}
\item Consequences of the Multiplication Axioms.
	\begin{enumerate}
		\item There is only one multiplicative identity.
		\item For $x\neq 0$, there is only one reciprocal $x^{-1}$
		\item $a\in R \setminus 0$, $a\cdot b$ has a unique solution $x = b\cdot a^{-1}$
	\end{enumerate}
\item Consequences of the Multiplication Axioms.
	\begin{enumerate}
		\item There is only one multiplicative identity.
		\item For $x\neq 0$, there is only one reciprocal $x^{-1}$
		\item $a\in R \setminus 0$, $a\cdot b$ has a unique solution $x = b\cdot a^{-1}$
	\end{enumerate}
\item Consequences of the Axiom Connecting Addition and Multiplication.
	\begin{enumerate}
		\item $x\cdot 0 = 0$
		\item $(xy=0)\implies (x=0)\vee(y=0)$
		\item $-x = -1 \cdot x$
		\item $-1 \cdot (-x) = x$
		\item $(-x)(-x) = x\cdot x$
	\end{enumerate}
\item Consequences of the Order Axioms
	\begin{enumerate}
		\item only one of the 3 relation hold for a particular $x$ and $y$
		\[ x<y, x=y, y>y \]
		\item \[(x<y)\wedge(y\leq z) \implies (x<z)\]
			\[(x\leq y)\wedge(y<z) \implies (x<z)\]
	\end{enumerate}
\item Consequences of the Axioms Connecting Order with Addition and Multiplication
	\begin{enumerate}
		\item 
		\[(x<y)\implies (x+z) < (y+z)\]
		\[(0<x)\implies (-x<0)\]
		\[(x\leq y) \wedge (z\leq w) \implies (x+z) \leq (y+w)\]
		\[(x\leq y) \wedge (z< w) \implies (x+z) < (y+w)\]
		\item
		\[(0<x) \wedge (0<y) \implies (0<xy)\]
		\[(x<0) \wedge (y<0) \implies (0<xy)\]
		\[(x<0) \wedge (0<y) \implies (xy<0)\]
		\[(0<x) \wedge (y<0) \implies (xy<0)\]
		\[(x<y) \wedge (0<z) \implies (xz<yz)\]
		\[(x<y) \wedge (z<0) \implies (yz<xz)\]
		\item \[0<1\]
		\item 
		\[(0<x)\implies (0<x^{-1})\]
		\[(0<x)\wedge(x<y) \implies (0<y^{-1}) \wedge (y^{-1} < x^{-1})\]
	\end{enumerate}
\end{enumerate}
\begin{definition}
$\{x| x\in \R \wedge 0<x\}$ is the set of \textbf{positive numbers}
\end{definition}
\begin{definition}
$\{x| x\in \R \wedge x<0\}$ is the set of \textbf{negative numbers}
\end{definition}

\subsection{The completeness Axiom and the Existence of a Least Upper (or Greatest Lower) Bound of a Set of Numbers}
\begin{definition}
A set $X\subset \R$ is said to be \textbf{bounded above} (resp. \textbf{bounded below}) if there exist a number $c \in \R$ such that $x\leq c$(resp. $c\leq x$) for all $x\in X$. $c$ in this case is the \textbf{upper bound} (resp. \textbf{lower bound}) of the set $X$. It's also called a \textbf{majorant} (resp. \textbf{minorant}) of X.
\end{definition}

\begin{definition}
A set that is bounded both above and below is called \textbf{bounded}
\end{definition}

\begin{definition}
An element $a \in X$ is called the \textbf{largest} or \textbf{maximal} (resp. \textbf{smallest} or \textbf{minimal}) element of $X$ is $x \leq a$ (resp. $a\leq x$) for all $x\in X$.
\end{definition}
\begin{definition}
\[
(a = \max X) := (a \in X \wedge \forall x \in X (x\leq a))
\]
\end{definition}
\begin{definition}
\[
(a = \min X) := (a \in X \wedge \forall x \in X (a\leq x))
\]
\end{definition}
\begin{definition}
The smallest number that bounds a set $X \subset \R$ from above is called the \textbf{least upper bound} of $X$ and denoted $\sup X$ (read "the supremum of X")
\[
(s = \sup X) := \forall x \in X ( (x\leq s) \wedge (\forall s' < s \exists x' \in X (s'<x')))
\]
\end{definition}

\begin{definition}
\[
(i = \sup X) := \forall x \in X ( (i\leq x) \wedge (\forall i' > i \exists x' \in X (x'<i')))
\]
\end{definition}

\begin{lemma}[The least upper bound principle]
Every nonempty set of real numbers that is bounded from above has a unique least upper bound.
\end{lemma}

\begin{lemma}
Every nonempty set of real numbers that is bounded from below has a unique greatest lower bound.
\end{lemma}

\section{The Most Important Classes of Real Numbers and Computational Aspects of Operations with Real Numbers}

\end{document}
\theend
