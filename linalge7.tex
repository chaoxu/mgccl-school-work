\documentclass[letter]{article}
\usepackage{amsmath}
\usepackage{amssymb}
\usepackage{amsfonts}
\usepackage{amsthm}
\usepackage{latexsym}
\usepackage{mathrsfs}
\usepackage{eufrak}
\usepackage[pdftex]{graphicx}
\usepackage{color}

\setlength{\topmargin}{-0.5in} \setlength{\textwidth}{6.5in}
\setlength{\oddsidemargin}{0.0in} \setlength{\textheight}{9.1in}

\newlength{\pagewidth}
\setlength{\pagewidth}{6.5in} \pagestyle{empty}

\newcommand{\R}{\mathbb{R}}
\newcommand{\N}{\mathbb{N}}
\newcommand{\B}{\mathfrak{B}}
\newcommand{\U}{\mathfrak{U}}
\newcommand{\Z}{\mathbb{Z}}
\newcommand{\im}{\operatorname{im}}
\newcommand{\image}{\operatorname{image}}
\newcommand{\rref}{\operatorname{rref}}
\newcommand{\rank}{\operatorname{rank}}
\newcommand{\nullity}{\operatorname{nullity}}
\newcommand{\tr}{\operatorname{tr}}
%\newcommand{\ker}{\operatorname{ker}}
\newtheorem{theorem}{Theorem}[section]
\newtheorem{lemma}[theorem]{Lemma}
\newtheorem{proposition}[theorem]{Proposition}
\newtheorem{corollary}[theorem]{Corollary}

\newenvironment{definition}[1][Definition]{\begin{trivlist}
\item[\hskip \labelsep {\bfseries #1}]}{\end{trivlist}}
\newenvironment{example}[1][Example]{\begin{trivlist}
\item[\hskip \labelsep {\bfseries #1}]}{\end{trivlist}}
\newenvironment{remark}[1][Remark]{\begin{trivlist}
\item[\hskip \labelsep {\bfseries #1}]}{\end{trivlist}}

\title{Linear Algebra(Bretscher) Chapter 7 Notes}
\date{}

\begin{document}
\maketitle
\vspace{-.5in}
\section{Dynamical Systems and Eigenvectors: An Introductory Example}
\subsection{Eigenvectors and Eigenvalues}
\begin{definition}
Consider an $n \times n$ matrix $A$. A nonzero vector $\vec v$ in $\R^n$ is called an \textbf{eigenvector} of $A$ if $A\vec v$ is a scalar multiple of $vec v$, that is, if
\[
A\vec v = \lambda \vec v
\]
for some scalar $\lambda$. This scalar may be zero.
The scalar $\lambda$ is called the \textbf{eigenvalue} associated with the eigenvector $\vec v$.
\end{definition}

\begin{theorem}
The possible real eigenvalues of an orthogonal matrix are $1$ and $-1$.
\end{theorem}

\subsection{Dynamical Systems and Eigenvectors}
\begin{theorem}[Discrete dynamical system]
Consider the dynamical system
\[
\vec x(t+1) = A\vec x(t) with \vec x(0) = \vec x_0
\]
Then
\[\vec x(t) = A^t\vec x_0\]
suppose that we can find a basis
\[
\vec v_1, \vec v_2, \ldots, \vec v_n
\]
of $\R^n$
consisting of eigenvectors of $A$, with
\[ A\vec v_n = \lambda_n \vec v_n\]
Find the coordinates $c_1, c_2, \ldots, c_n$ of vector $\vec x_0$ with respect to basis $\vec v_1, \vec v_2, \ldots, \vec v_n$:\\
\[
\vec x_0 = c_1\vec v_1 + c_2\vec v_2 + \ldots + c_n\vec v_n
\]
Then
\[
\vec x(t) = c_1\lambda_1^t\vec v_1 + c_2\lambda_2^t\vec v_2 + \ldots + c_n\lambda_n^t\vec v_n 
\]
\end{theorem}

\begin{definition}
Consider a discrete dynamic system
\[\vec x(t+1) = A\vec x(t)\]
initial value $\vec x(0) = \vec x_0$,
where $A$ is a $2\times 2$ matrix. In this case, the state vector $\vec x(t) = \begin{bmatrix}
x_1(t)\\
x_2(t)
\end{bmatrix}$
 can be represented geometrically in the $x_1$-$x_2$-plane.
The endpoints of state vectors $vec x(0) = \vec x_0, vec x(1) = A\vec x_0,\ldots$ form the \textbf{discrete trajectory} of this system, representing its evolution in the future.
A \textbf{discrete phase portrait} of the system $\vec x(t+1) = A\vec x(t)$ shows trajectories for various initial states, capturing all the qualitatively different scenarios.
\end{definition}

\section{Finding the Eigenvalues of a Matrix}
\begin{theorem}
Consider an $n \times n$ matrix $A$ and a scalar $\lambda$. Then $\lambda$ is an eigenvalue of $A$ if and only if
\[
\det(A-\lambda I_n) = 0.
\]
This is called the \textbf{characteristic equation}(or the \textbf{secular equation}) of matrix A.
\end{theorem}

\begin{theorem}
The eigenvalues of a triangular matrix are its diagonal entries.
\end{theorem}

\begin{definition}
The sum of the diagonal entries of a square matrix $A$ is called the \textbf{trace} of $A$, denoted by $\tr A$
\end{definition}

\begin{theorem}
\[
\det(A - \lambda I_2) = \lambda^2 - (\tr A)\lambda + \det A = 0
\]
\end{theorem}

\begin{theorem}
If $A$ is an $n \times n$ matrix, then $\det(A-\lambda I_n)$ is a polynomial of degree $n$, of the form
\[
(-\lambda)^n + (\tr A)(-\lambda)^{n-1} + \ldots + \det A\\
= (-1)^n\lambda^n + (-1)^{n-1}(\tr A)\lambda^{n-1} + \ldots + \det A
\]
This is called the \textbf{characteristic polynomial} of $A$, denoted by $f_A(\lambda)$.
\end{theorem}

\begin{definition}
We say that an eigenvalue $\lambda_0$ of a square matrix $A$ has \textbf{algebraic multiplicity $k$} if $\lambda_0$ is a root of multiplicity $k$ of the characteristic polynomial $f_A(\lambda)$, meaning that we can write
\[
f_A(\lambda) = (\lambda_0 - \lambda)^k g(\lambda)
\]
for some polynomial $g(\lambda)$ with $g(\lambda_0)\neq 0$.
\end{definition}

\begin{theorem}
An $n\times n$ matrix has at most $n$ real eigenvalues, even if they are counted with their algebraic multiplicities.\\
If $n$ is odd, then an $n \times n$ matrix has at least one real eigenvalue. 
\end{theorem}

\begin{theorem}[Eigenvalues, determinant, and trace]
If an $n\times n$ matrix $A$ has the eigenvalues $\lambda_1, \lambda_2, \ldots, \lambda_n$, listed with their algebraic multiplicities, then
\[
\det A = \lambda_1\lambda_2\ldots\lambda_n
\]
and
\[
\tr A = \lambda_1 + \lambda_2 + \ldots + \lambda_n
\]
\end{theorem}
\section{Finding the Eigenvectors of a Matrix}
\begin{definition}
Consider an eigenvalue $\lambda$ of an $n\times n$ matrix $A$. Then the kernel of the matrix $A - \lambda I_n$ is called the \textbf{eigenspace} associated with $\lambda$, denoted by $E_\lambda$:
\[
E_\lambda = \ker(A - \lambda I_n) = \{\vec v in \R^n : A\vec v = \lambda \vec v\}
\]
\end{definition}

\begin{definition}
Consider an eigenvalue $\lambda$ of an $n \times n$ matrix $A$. The dimension of eigenspace $E_ \lambda = \ker(A - \lambda I_n)$ is called the \textbf{geometric multiplicity} of eigenvalue $\lambda$. Thus, the geometric multiplicity is the nullity of matrix $A - \lambda I_n$, or $n-\rank(A - \lambda I_n)$.
\end{definition}

\begin{definition}
Consider an $n\times n$ matrix $A$. A basis of $\R^n$ consisting of eigenvectors of $A$ is called an \textbf{eigenbasis} for $A$.
\end{definition}

\begin{theorem}[Eigenbases and geometric multiplicities]
\begin{enumerate}
\item Consider an $n\times n$ matrix $A$. If we find a basis of each eigenspace of $A$ and concatenate all these bases, then the resulting eigenvecotrs $vec v_1, \ldots \vec v_s$ will be linearly independent. (Note that $s$ is the sum of the geometric multiplicities of the eigenvalues of $A$.)
\item There exist an eigenbasis for an $n\times n$ matrix $A$ if (and only if) the geometric multiplicities of the eigenvalues add up to $n$ (meaning that $s = n$)
\end{enumerate}
\end{theorem}

\subsection{Eigenvalues and Similarity}
\begin{theorem}[The eigenvalues of similar matrices]
Suppose matrix $A$ is similar to $B$. Then \\
\begin{enumerate}
\item Matrices $A$ and $B$ have the same characteristic polynomial, that is $f_A(\lambda) = f_B(\lambda)$.
\item $\rank(A) = \rank(B)$ and $\nullity(A) = \nullity(B)$
\item Matrices $A$ and $B$ have the same eigenvalues, with the same algebraic and geometric multiplicities.
\item Matrices $A$ and $B$ have the same determinant and the same trace: $\det A = \det B$ and $\tr A = \tr B$
\end{enumerate}
\end{theorem}

\begin{theorem}
If $\lambda$ is an eigenvalue of a square matrix $A$, then\\
(geometric multiplicity of $\lambda) \leq ($algebraic multiplicity of $\lambda$)
\end{theorem}

\section{Diagonalization}
\begin{theorem}[The matrix of a linear transformation with respect to an eigenbasis]
Consider a linear transformation $T(\vec x) = A \vec x$, where $A$ is a square matrix. Suppose $\mathfrak{D} = (\vec v_1,\vec v_2,\ldots,\vec v_n)$ is an eigenbasis for $T$, with $A\vec v_i = \lambda_i\vec v_i$. Then the $\mathfrak{D}$-matrix $D$ of $T$ is
\[
D = S^{-1}AS = 
\begin{bmatrix}
\lambda_1 & 0 & \ldots & 0\\
0 & \lambda_2 & \ldots & 0\\
\vdots & \vdots &\ddots &\vdots\\
0 & 0 &\ldots & \lambda_n 
\end{bmatrix}
\]
where
\[
S = 
\begin{bmatrix}
\vec v_1 & \vec v_2 & \ldots & \vec v_n
\end{bmatrix}
\]
Matrix $D$ is diagonal, and its diagonal entries are the eigenvalues $\lambda_1, \lambda_2, \ldots, \lambda_n$ of $T$.
\end{theorem}

\begin{definition}
An $n\times n$ matrix $A$ is called \textbf{diagonalizable} if $A$ is similar to some diagonal matrix $D$, that is, if there exists an invertible $n\times n$ matrix $S$ such that $S^{-1}AS$ is diagonal.
\end{definition}

\begin{theorem}[Eigenbases and diagonalization]
\begin{enumerate}
\item Matrix $A$ is diagonalizable if (and only if) there exists and eigenbasis for $A$.
\item If an $n\times n$ matrix $A$ has $n$ distinct eigenvalues, then $A$ is diagonalizable.
\end{enumerate}
\end{theorem}

\begin{theorem}[Diagonalization]
Suppose we are asked to determine whether a given $n\times n$ matrix $A$ is diagonalizable. If so, we wish to find an invertible matrix $S$ such that $S^{-1}AS$ is diagonal.
We can proceed as follows.
\begin{enumerate}
\item Find the eigenvalues of $A$, that is, solve the characteristic equation $f_A(\lambda) = \det(A - \lambda I_n) = 0$
\item For each eigenvalue $\lambda$, find a basis of the eigenspace $E_\lambda = \ker(A - \lambda I_n)$.
\item Matrix $A$ is diagonalizable if (and only if) the dimensions of the eigenspaces add up to $n$. In this case, we find an eigenbasis $\vec v_1, \vec v_2, \ldots, \vec v_n$ by concatenating the bases of the eigenspaces we found in last step. Let $S = \begin{bmatrix}
\vec v_1& \vec v_2 & \ldots & \vec v_n
\end{bmatrix}$. Then matrix $S^{-1}AS = D$ is diagonal, and the $i$th diagonal entry of $D$ is the eigenvalue $\lambda_i$ associated with $\vec v_i$.
\end{enumerate}
\end{theorem}

\subsection{Powers of a Matrix}
\begin{theorem}[Powers of a diagonalizable matrix]
To compute the powers $A^t$ of a diagonalizable matrix $A$, first diagonalize A. Then $A = SDS^{-1}$ and $A^t = SD^tS^{-1}$.  To compute $D^t$, raise the diagonal entries of $D$ to the $t$th power. 
\end{theorem}

\subsection{The Eigenvalues of a Linear Transformation}
\begin{definition}
Consider a linear transformation $T$ from $V$ to $V$, where $V$ is a linear space. A scalar $\lambda$ is called an \textbf{eigenvalue} of $T$ if there exists a nonzero element $f$ of $V$ such that
\[
T(f) = \lambda f
\]
Such an $F$ is called an \textbf{eigenfunction} if $V$ consists of functions, and \textbf{eigenmatrix} if $V$ consists of matrices, and so on. In theoretical work, the inclusive term \textbf{eigenvector} is often used for $f$. \\
Suppose that $V$ is finite dimensional. Then a basis $\mathfrak{D}$ of $V$ consisting of eigenvectors of $T$ is called an \textbf{eigenbasis} for $T$. We say that transformation $T$ is \textbf{diagonalizable} if the matrix $T$ with respect to some basis is diagonal.
\end{definition}
\end{document}
\theend