\documentclass[letter]{article}
\usepackage{amsmath}
\usepackage{amssymb}
\usepackage{amsfonts}
\usepackage{amsthm}
\usepackage{latexsym}
\usepackage{mathrsfs}
\usepackage{eufrak}
\usepackage[pdftex]{graphicx}
\usepackage{color}

\setlength{\topmargin}{-0.5in} \setlength{\textwidth}{6.5in}
\setlength{\oddsidemargin}{0.0in} \setlength{\textheight}{9.1in}

\newlength{\pagewidth}
\setlength{\pagewidth}{6.5in} \pagestyle{empty}

\newcommand{\R}{\mathbb{R}}
\newcommand{\N}{\mathbb{N}}
\newcommand{\B}{\mathfrak{B}}
\newcommand{\U}{\mathfrak{U}}
\newcommand{\Z}{\mathbb{Z}}
\newcommand{\im}{\operatorname{im}}
\newcommand{\image}{\operatorname{image}}
\newcommand{\rref}{\operatorname{rref}}
\newcommand{\rank}{\operatorname{rank}}
\newcommand{\nullity}{\operatorname{nullity}}
%\newcommand{\ker}{\operatorname{ker}}
\newtheorem{theorem}{Theorem}[section]
\newtheorem{lemma}[theorem]{Lemma}
\newtheorem{proposition}[theorem]{Proposition}
\newtheorem{corollary}[theorem]{Corollary}

\newenvironment{definition}[1][Definition]{\begin{trivlist}
\item[\hskip \labelsep {\bfseries #1}]}{\end{trivlist}}
\newenvironment{example}[1][Example]{\begin{trivlist}
\item[\hskip \labelsep {\bfseries #1}]}{\end{trivlist}}
\newenvironment{remark}[1][Remark]{\begin{trivlist}
\item[\hskip \labelsep {\bfseries #1}]}{\end{trivlist}}

\title{Linear Algebra(Bretscher) Chapter 4 Notes}
\date{}

\begin{document}
\maketitle
\vspace{-.5in}

\section{Introduction to Linear Spaces}
\begin{definition}
A \textbf{linear space} $V$ is a set with the following rules for all $f,g,h \in V$ and $c,k\in \R$.
\begin{enumerate}
\item $(f+g)+h=f+(g+h)$
\item $f+g = g+f$
\item There exist a unique 0, such that $f+0 = f$
\item There exist a unique $g\in V$ such that $f+g = 0$, $g$ is also $-f$.
\item $k(f+g) = kf + kg$
\item $(c+k)f = cf + kf$
\item $c(kf) = (ck)f$
\item $1f = f$
\end{enumerate}
\end{definition}
\begin{definition}
A subset $W$ of a linear space $V$ is called a \textbf{subspace} of $V$ if
\begin{enumerate}
\item $W$ contains 0 of $V$.
\item $W$ is closed under addition.
\item $W$ is closed under scalar multiplication.
\end{enumerate}
\end{definition}

Elements $f_1,\ldots,f_n$ in $V$\\
\begin{definition}
$f_1,\ldots,f_n$ \textbf{span} $V$ if every $f$ in $V$ can be expressed as a linear combination of $f_1,\ldots,f_n$.
\end{definition}

\begin{definition}
$f_i$ is \textbf{redundant} if it's a linear combination of $f_1,\ldots,f_{i-1}$. If non of them are redundant, $f_1,\ldots,f_n$ is called \textbf{linearly independent}.
\end{definition}

\begin{definition}
$f_1,\ldots,f_n$ is a \textbf{basis} of $V$ if they span $V$ and are linearly independent. For any $f = c_1f_1+\ldots+c_nf_n$. The coefficient $c_1,\ldots c_n$ are called the \textbf{coordinates} of $f$ with respect to the basis $\B = (f_1,\ldots,f_n)$. The vector 
\[\begin{pmatrix}
c_1\\
\vdots\\
c_{n}
\end{pmatrix}\]
is the \textbf{$\B$-coordinate vector} of $f$, denoted by $[f]_{\B}$.\\
The transformation
\[
L(f) = [f]_{\B} = \begin{pmatrix}
c_1\\
\vdots\\
c_{n}
\end{pmatrix}\
\]
from $V$ to $\R^n$ is called the \textbf{$\B$-coordinate transformation}, denoted by $L_\B$.
\end{definition}

\begin{theorem}[Linearity of the coordinate transformation $L_\B$]
If $\B$ is a basis of a linear space $V$, then
\begin{enumerate}
\item $[f+g]_\B = [f]_\B + [g]_\B$
\item $[kf]_\B = k [f]_\B$
\end{enumerate}
\end{theorem}

\begin{theorem}
If a linear space $V$ has a basis of $n$ elements, all other bases of $V$ consist of $n$ elements. and $n$ is the \textbf{dimension} of $V$:
\[\dim(V) = n\]
\end{theorem}

\begin{theorem}[Find a basis of a linear space $V$]
It can be done with following steps
\begin{enumerate}
\item Write down a element of $V$ in terms of arbitrary constants.
\item Express it as coefficients of linear combination of some elements of $V$, where coefficients are the arbitrary constants.
\item Verify that the elements of $V$ in the linear combination are linearly independent. If yes, they form a basis of $V$.
\end{enumerate}
\end{theorem}

\begin{theorem}
The solution of $n$th-order linear differential equation with constant coefficients form a $n$-dimensional subspace of $\mathbb{C}^n$.
\end{theorem}
\begin{definition}
A linear space $V$ is called \textbf{finite dimensional} if $\dim(V) = n$ for some integer $n$. Else it's \textbf{infinite dimensional}.
\end{definition}

\section{Linear Transformations and Isomorphisms}
\begin{definition}
$T:V\to W$ is a \textbf{linear transformation} if
\[T(f+g) = T(f)+T(g)\]
\[T(kf) = kT(f)\]
$f,g\in V, k\in \R$. Let
\[\im(T) = \{T(f):f\in V\}\]
\[\ker(T) = \{f\in V: T(f) = 0\}\]
$\dim(\im(T))$ is called the \textbf{rank} of T.\\
$\dim(\ker(T))$ is called the \textbf{nullity} of T.\\
and
$\dim(V) = \rank(T) + \nullity(T)$
\end{definition}

\begin{definition}
An invertible linear transformation $T$ is called an \textbf{isomorphism}. $V$ is isomorphic to $W$ if there exist an isomorphism $T:V\to W$.
\end{definition}

\begin{theorem}[Coordinate transformations are isomorphisms]
If $\B = (f_1,\ldots,f_n)$ is a basis of $V$, then coordinate transformation $L_\B(f) = [f]_\B$ from $V$ to $\R^n$ is an isomorphism. $V$ is isomorphic to $\R^n$. Or, any $n$-dimensional linear space $V$ is isomorphic to $\R^n$.
\end{theorem}

\begin{theorem}[Properties of isomorphisms]
\begin{enumerate}
\item $T:V\to W$ is an isomorphism iif $\ker(T) = \{0\}$ and $\im(T) = W$.
\item If $V$ is isomorphic to $W$, then $\dim(V) = \dim(W)$.
\item Let $T:V\to W$ be an linear transformation and $\ker(T) = \{0\}$. If $\dim(V) = \dim(W)$, then $T$ is a isomorphism.
\item Let $T:V\to W$ be an linear transformation and $\im(T) = W$. If $\dim(V) = \dim(W)$, then $T$ is a isomorphism.
\end{enumerate}
\end{theorem}

\begin{theorem}
A procedure checks if $V$ and $W$ are isomorphic.
\begin{enumerate}
\item Is $\dim(V) = \dim(W)$? no return false.
\item If you can write a formula for the inverse of $T$, return true. else go to 3.
\item Is $\ker(T) = \{0\}$, yes return true, no return false, else go to 4.
\item Is $\im(T) = W$, yes return true, no return false.
\end{enumerate}
\end{theorem}


\section{The Matrix of a Linear Transformation}
\begin{definition}
Consider a linear transformation $T$ from $V$ to $V$, where $V$ is an $n$-dimensional linear space. Let $\B$ be a basis of $V$. Consider the linear transformation $L_\B \circ T \circ L_\B^{-1}$ from $\R^n$ to $\R^n$, with standard matrix $B$, meaning that $B\vec x = L_\B(T(L_\B^{-1}(\vec x)))$ for all $\vec x$ in $\R^n$. This matrix $B$ is called the \textbf{$\B$-matrix of transformation $T$}.
\end{definition}

\begin{theorem}
Consider a linear transformation $T$ from $V$ to $V$, and let $B$ be the matrix of $T$ with respect to a basis $\B = (f_1,\ldots,f_n)$ of $V$. Then
\[
B = \begin{bmatrix}
[T(f_1)]_\B & \cdots & [T(f_n)]_\B
\end{bmatrix}
\]
The columns of $B$ are the $\B$-coordinate vectors of the transforms of the basis elements $f_1,\ldots,f_n$ of $V$.
\end{theorem}

\subsection{Change of Basis}
\begin{definition}
Consider two bases $\U$ and $\B$ of an $n$-dimensional linear space $V$. Consider the linear transformation $L_\U \circ L_\B^-1$ from $\R^n$ to $\R^n$ , with standard matrix $S$, meaning that $S\vec x = L_\U(L_\B^-1(\vec x))$ for all $\vec x$ in $R^n$. This invertible matrix $S$ is called the \textbf{change of basis matrix} from $\B$ to $\U$, sometimes denoted by $S_{\B\to \U}$
\end{definition}

\begin{theorem}[Change of basis in a subspace of $\R^n$]
Consider a subspace $V$ of $\R^n$ with two bases $\U = (\vec a_1, \ldots, \vec a_m)$ and $\B = (\vec b_1, \ldots, \vec b_m)$. Let $S$ be the change of basis matrix from $\B$ to $\U$. Then the following equation holds:
\[
\begin{bmatrix}
\vec b_1& \cdots & \vec b_m
\end{bmatrix}
= 
\begin{bmatrix}
\vec a_1& \cdots & \vec a_m
\end{bmatrix}S
\]
\end{theorem}

\begin{theorem}[Change of basis for the matrix of a linear transformation]
Let $V$ be a linear space with two given base $\U$ and $\B$. Consider a linear transformation $T$ from $V$ to $V$, and let $A$ and $B$ be the $\U$- and the $\B$-matrix of $T$, respectively. Let $S$ be the change of basis matrix from $\B$ to $\U$. Then $A$ is similar to $B$, and
\[AS = SB \text{ or } A = SBS^{-1} \text{ or } B = S^{-1}AS\]
\end{theorem}
\end{document}


\theend
