\documentclass[letter]{article}
\usepackage{amsmath}
\usepackage{amssymb}
\usepackage{amsfonts}
\usepackage{latexsym}
\usepackage{amsthm}
\usepackage[pdftex]{graphicx}
\usepackage{color}
\usepackage{xspace}
\usepackage[x11names, rgb]{xcolor}
\usepackage[utf8]{inputenc}
\usepackage{tikz}
\usetikzlibrary{snakes,arrows,shapes}

\setlength{\topmargin}{-0.5in} \setlength{\textwidth}{6.5in}
\setlength{\oddsidemargin}{0.0in} \setlength{\textheight}{9.1in}

\newlength{\pagewidth}
\setlength{\pagewidth}{6.5in} \pagestyle{empty}

\newcommand{\R}{\mathbb{R}}
\newcommand{\N}{\mathbb{N}}
\newcommand{\Z}{\mathbb{Z}}
\newtheorem{lma}{Lemma}
%\author{Chao Xu}
\title{MAT 205 stuff}
\date{}
\begin{document}
\maketitle
\vspace{-.5in}
It's possible for me to make a mistake. the general way to solve the problem
should hold. and I can't do one of the sub-problem.
\section{}

\subsection*{a)}
$\nabla f(x,y) = (\frac{\partial f}{\partial x},\frac{\partial f}{\partial y})=
(\cos \frac{\pi}{4}, \sin \frac{\pi}{4}) =
(\frac{1}{\sqrt{2}},\frac{1}{\sqrt{2}})$
\subsection*{b)}
$(1,-1),(-1,1)$
\subsection*{c)}
$(\frac{\partial f}{\partial x},\frac{\partial f}{\partial y}) {1 \choose 2} =
\frac{1}{\sqrt{2}}+2 \frac{1}{\sqrt{2}} = \frac{3}{\sqrt{2}}$
\subsection*{d)}
$f(x,y,z) = x^2+2xy+z$.
$\nabla f(x_0)$ is the normal vector.
$(2x+2y,2x,1) = (4,2,1)$

\section{}
Should be 0. $T(x,y) = (\frac{1}{4}x, 0) + (0, \frac{1}{2}y) =
(u,v)$. $\int_{T(R)} xy dx dy$. can be transformed into $ \int_R uv det|T| du dv
= \frac{1}{8} \int_R uv dudv $. $R$ is a circle. For every $(u,v)$, there exist
$(-u,v)$, we have $\int_{semicircle} (uv -uv) dudv= 0$.

\section{}
Langrange multiplier.
$f(x,y) = (x-1)^2+(y-2)^2$\\
Instead of $x^2 + y^2 \leq 45$, we have $x^2 + y^2 = 45$. $f(x,y)$ can't reach
as large value when it's less than 45. $g(x,y) = x^2+y^2-45$\\
$f'(x) + \lambda g'(x) = 0$\\
$(x-1)^2+(y-2)^2+x^2+y^2-45=0$\\
Critical points
$2x-2+2\lambda x =0$ and $2y-4 + 2\lambda y =0$\\
$(1+\lambda)x = 1$ and $(1+\lambda)y = 2$\\
$y=2x$\\
$x^2+(2x)^2=45$\\
$x = \pm 3$\\
$y = \pm 6$\\
when $x=-3$, $y=-6$, $f(x,y)$ takes on max value of 80.

\section{}
$F((u,v),(x,y)) = (u^2xy^2+uvx+y-2,vx^2+u^2y^2+uv-5)$
$F((1,2),(x,y)) = (xy^2+2x+y-2,2x^2+y^2-3)$
$F_y((1,2),(1,-1)) = \begin{pmatrix}y^2+2 & 2xy+y\\ 4x & 2y\end{pmatrix} =
\begin{pmatrix}3 & -3\\ 4 & -2\end{pmatrix}$
$\begin{pmatrix}3 & -3\\ 4 & -2\end{pmatrix}$ has a inverse.
By inverse function theorem, it's possible to solve the equation locally near
u=1 and v=2.
I can't find the partial.

\subsection*{not related above}

$f(g(2,3)) = f'(g(2,3))g'(2,3)$\\
$=f'(6,11)g'\begin{pmatrix}v & u\\ 1 & 2v\end{pmatrix}_{(2,3)}$\\
$= \begin{pmatrix}2 & 6\\ 3 & 11\end{pmatrix}
\begin{pmatrix}3 & 2\\ 1 &
4\end{pmatrix}$
\end{document}
\theend
