\documentclass[letter]{article}
\usepackage{amsmath}
\usepackage{amssymb}
\usepackage{amsfonts}
\usepackage{amsthm}
\usepackage{latexsym}
\usepackage{mathrsfs}
\usepackage{eufrak}
\usepackage[pdftex]{graphicx}
\usepackage{color}
\usepackage{mathrsfs}
\usepackage[inline]{asymptote}
\setlength{\topmargin}{-0.5in} \setlength{\textwidth}{6.5in}
\setlength{\oddsidemargin}{0.0in} \setlength{\textheight}{9.1in}

\newlength{\pagewidth}
\setlength{\pagewidth}{6.5in} \pagestyle{empty}

\newcommand{\R}{\mathbb{R}}
\newcommand{\N}{\mathbb{N}}
\newcommand{\B}{\mathfrak{B}}
\newcommand{\Z}{\mathbb{Z}}
\newtheorem{theorem}{Theorem}[section]
\newtheorem{lemma}[theorem]{Lemma}
\newtheorem{proposition}[theorem]{Proposition}
\newtheorem{corollary}[theorem]{Corollary}

\newenvironment{definition}[1][Definition]{\begin{trivlist}
\item[\hskip \labelsep {\bfseries #1}]}{\end{trivlist}}
\newenvironment{example}[1][Example]{\begin{trivlist}
\item[\hskip \labelsep {\bfseries #1}]}{\end{trivlist}}
\newenvironment{remark}[1][Remark]{\begin{trivlist}
\item[\hskip \labelsep {\bfseries #1}]}{\end{trivlist}}
\begin{asydef}
usepackage("bm");
texpreamble("\def\V#1{\bm{#1}}");
\end{asydef}

\title{CSE 555 Spring 2010 HW \#1}
\date{}
\author{Chao Xu}
\begin{document}
\maketitle
\vspace{-.5in}
\section*{(1)}
If point $a,b$ has coordinate $(a_x,a_y)$ and $(b_x,b_y)$, then \[
 BB(\{a,b\})
= \{(x,y)| \min(a_x,b_x) \leq x\leq \max(a_x,b_x) \wedge \min(a_y,b_y) \leq
y\leq
\max(a_y,b_y)\}     
                                                                \]
The definition can be generalized so for $n$ points $S = \{p_1,p_2,\ldots p_n\}$
with
coordinates $S_x$ and $S_y$, then 
\[
 BB(S)
= \{(x,y)| \min(S_x) \leq x\leq \max(S_x) \wedge \min(S_y) \leq
y\leq \max(S_y)\}      
                                                              \]
$BB(S)$ contains all the points in $S$ and contains
all $BB({p_i,p_j})$.
\[
RCH(S)\subseteq BB(S)
\]
\[
\bigcup_{i,j} BB(\{p_i,p_j\}) \subseteq RCH(S)
\]
wlog, let $a, b, c, d$ be points
with the largest $x$ coordinate, largest $y$, coordinate, smallest $x$
coordinate, and smallest $y$ coordinate respectively.\\
\[
BB(\{a,b\}) = \{(x,y)| b_x \leq x\leq \max(S_x) \wedge a_y \leq y\leq
\max(S_y)\}
\]
\[
BB(\{b,c\}) = \{(x,y)| \min(S_x) \leq x\leq b_x \wedge c_y \leq y\leq
\max(S_y)\}
\]
\[
BB(\{c,d\}) = \{(x,y)| \min(S_x) \leq x\leq d_x \wedge \min(S_y) \leq y\leq
c_y\}
\]
\[
BB(\{d,a\}) = \{(x,y)| d_x \leq x\leq \max(S_y) \wedge \min(S_y) \leq y\leq
a_y\}
\]
Which shows 
\[BB(\{a, b\})\cup BB(\{b, c\}) \cup BB(\{c, d\}) \cup BB(\{d,a\})= BB(S)\]
\[BB(S) \subseteq \bigcup_{i,j} BB(\{p_i,p_j\}) \subseteq RCH(S)\]
\[BB(S) \subseteq RCH(S) \subseteq BB(S)\]
Then \[RCH(S) = BB(S)\]

The above holds even when $|\{a,b,c,d\}|< 4$.\\
There exist a $O(n)$ algorithm to find
$\max(S_x),\max(S_y),\min(S_x),\min(S_x)$, 
Which is enough information to define $BB(S)$. $RCH(S)$ can be computed in
$O(n)$ time.

\section*{Extra Credit}
\begin{center}
\begin{asy}
size(8cm,0);
pen colour=black;

pair O=(0,0);
pair P=(0,1);
pair Q=(0,-1);
real r=1;
real r2 = sqrt(2);

path c1=circle(O,r);
pair A=(-r2/2,r2/2);
pair B=(r2/2,r2/2);
pair A2=(-r2/2,-r2/2);
pair B2=(r2/2,-r2/2);
pair C=(0,r2);
pair C2 = (0,-r2);
pair i1 = (0,r2/2);
pair i2 = (0,-r2/2);
draw(c1);
draw(A--B);
draw(A2--B2);
path c2 = circle((0,r2/2),r2/2);
path c3 = circle((0,-r2/2),r2/2);
path c4 = circle((0,0),r2);
draw(c2);
draw(c3);
draw(c4);
draw(C--C2);


dot(A);label("$A$",A,NW);
dot(B);label("$B$",B,NE);
dot(A2);label("$A'$",A2,SW);
dot(B2);label("$B'$",B2,SE);
dot(O);label("$O$",O,SE);
dot(P);label("$P$",P,NE);
dot(Q);label("$Q$",Q,SE);
dot(C);label("$C$",C,N);
dot(C2);label("$C'$",C2,S);
dot(i1);label("$O'$",i1,SE);
dot(i2);label("$O''$",i2,NE);
\end{asy}
\end{center}

Let $P,Q\in S$, then $MEB(\{P,Q\}) \subseteq CCH(S)$.
Consider a process that will create a set $MEB(\{C,C'\})$ where $PQ\subset CC'$. Illustrated above. Find $A,B$ on $MEB(\{P,Q\})$ such that $AB \leq \sqrt{2}PQ$ and the midpoint $O'$ of $AB$ lies on $PQ$. Find the other pair $A',B'$ and $O''$ with the same property. Draw circle $O',O''$ with radius $AB/2$. Extend $PQ$ such that it intersect circle $O'$ and $O''$ at point $C$ and $C'$. I claim $PQ\subset CC'$. and $MEB(\{P,Q\})\subset MEB(\{C,C'\})$\\
$MEB(\{C,C'\})\subseteq CCH(S)$ by definition.\\
The above process can be repeated indefinitely, thus $CCH(S)$ contains a circle of infinite radius. $CCH(S)$ is the plane when $|S|\geq 2$.

\section*{(3)}
There exist a $O(n \log n)$ algorithm. \\
Select a point $c_1$ on the convex hull
(Point with the largest $x$-coordinate, if there is more than one, the point
with
largest $y$-coordinate).\\
Sort $p_i$ by the slope of $c_1p_i$.\\
Construct the convex hull. Let the points on the convex hull be $c_i$, ordered
by
the slope of $c_1c_i$. For all $p_k$ such that $c_1p_k$'s slope are between
$c_1c_i$ and $c_1c_{i+1}$. Then $c_1, c_i$ and $c_{i+1}$ are the witness of
$p_k$.

\begin{proof}
Ray $c_1p_k$ lies between angle $c_ic_1c_{i+1}$. Implies $p_k$ doesn't lie in
triangle $c_ic_1c_{i+1}$
only if it's on the halfplane of $c_ic_{i+1}$ that doesn't contain $c_1$.
Suppose it
is true, then $p_k$ is outside the convex hull. Which is a contradiction because
$p_k$ is in the convex hull.
\end{proof}

\section*{(4)}
\subsection*{a}
Let $S$ and $T$ be two convex sets. The intersection is all points $p$ and $q$
in both $S$ 
and $T$. For any $p$ and $q$ in $S$, $pq$ is in $S$, 
same goes for $p$ and $q$ in $T$. For any two points in both $S$ and $T$, 
the segment formed by those points also lies in $S$ and $T$.
It implies $S \cap T$ is convex.

\subsection*{b}
Let $\mathcal{P}$ be the smallest perimeter polygon contains set of points $P$. If it's
non-convex. Suppose connect vertex $v_i$ and $v_{i+2}$, such that we have a
segment not in
$\mathcal{P}$, and $v_i,v_{i+1},v_{i+2}$ forms consecutive 2 edges of $\mathcal{P}$.
We have a new polygon with all vertices of $\mathcal{P}$ except $v_{i+1}$. It form a polygon
covering all points contained by $\mathcal{P}$, but has smaller parameter(by
triangle inequality).
A contradiction. The smallest polygon contains set of points $P$ must be convex.

\subsection*{c}
If $\mathcal{P}$ is a subset of some convex set, then that convex set contains
all
the points in $P$. Suppose convex set $P \subseteq C\subset \mathcal{P}$. Then there exist a point $p$
in 
$\mathcal{P}$ that's not in $C$. Partition the triangle by connect $v_1$ with
every
other vertice of $\mathcal{P}$. Point $p$ has to lie either on a triangle
partition
or inside a triangle partition of $\mathcal{P}$. Suppose it is on a edge of a
triangle
partition, then it lies on $v_iv_j$. $p$ must be in $C$ because $v_iv_j$ is
contained in
$C$. If $p$ lies inside a triangle $v_1v_iv_j$.
Ray $pv_1$ meets $v_iv_j$ at point $p'$. $p'$ is contained in $C$(due to above
result).
$p$ must lie in $C$ since it is a point of the segment $p'v_1$.
There can't be a point in $C$ that's not in $\mathcal{P}$. $\mathcal{P} \subseteq C$, a contradiction to $C \subset P$. For all convex
set $C$ containing points of $P$. $\mathcal{P}\subseteq C$.

\section*{(5)}
\subsection*{a}
Use sweep line algorithm from every direction. It stops when it touchs a point, and mark that point on the convex hull. If it touchs $k$ point at the same time, mark the largest segment from connecting 2 of those $k$ points as part of the convex hull. Each sweep line can either generate 1 point or straight line segments on the convex hull. If it generate 1 point, then it must be a point on the arc of a circle, because extreme points lies on the circles. The convex hull must consist of straight line segments and pieces of circles in $S$.

\subsection*{b}
\begin{lemma}
For the any point $P$ on $CH(S)$ . There exist a unit circle lies inside the $CH(S)$ such that its center is 1 unit apart from $P$.
\end{lemma}

\begin{proof}
If $P$ lies on the arc of a circle $C$, then $C$ is the unit circle that lies inside $CH(S)$.\\
If $P$ lies on the line segment $AB$ tagent to circles $C_1,C_2\in S$. Then there exist $A'B'$ on $C_1$ and $C_2$ such that is parallel to $AB$. $A'B'$ is inside $CH(S)$ because $A'$ and $B'$ lies inside $CH(S)$. Connect $ABB'A'$ we have a rectangle with width 2 that lies inside $CH(S)$. $P$ lies on one side, then draw $P'$ on the other side such that $PP' = 2$. Then there exist a unit circle centered at the midpoint of $PP'$ and lies inside $CH(S)$.
\end{proof}


Suppose we already have a convex hull for $CH(S-\{C\})$ where $C$ is a circle on the convex hull. One can add $C$ into $S$ to find a new convex hull.\\
If a circle can be on the convex hull for two times, then there must be a set of 4 points  $\{a,b,c,d\}$(ordered clockwise) on the circle, such that $a,b$ is on convex hull, $c,d$ is on convex hull and $b,c$, $d,a$ inside the convex hull.\\

$ad$ and $cd$ have to lie on a convex hull for $CH(S-\{C\})$. If $k$ is a point on $ad$, then it's not possible to find a unit circle that lies inside $CH(S-\{C\})$ and tangent to $k$. A contradiction to the lemma. Thus there can't be 2 pieces of the same unit circle lies on $CH(S)$. 

\subsection*{c}
If the circle $C$ centered at a vertex $v_i$ of $CH(S')$. Bisect angle $v_{i-1}v_iv_{i+1}$ that lies outside $CH(S')$ by ray $v_ip$, where $p$ is on $C$. $p$ is a extreme point, There exist a sweep line that touchs $p$ before touch any other point. Thus $C$ appear on the boundary of convex hull $CH(S)$.\\
Similar argument can be used to prove if $C$ is between two vertex $v_iv_{i+1}$. There is a sweep line touches $C$ while touch two circles that centers at vertex $v_i$ and $v_{i+1}$. \\
The boundary of $CH(S)$ are 1 unit away from $CH(S')$ (The rectangle constructed in the proof of the lemma is the proof). It proves the center of the circle lies inside $CH(S')$ can't be on the boundary of $CH(S)$ unless it have a radius greater than 1.

\subsection*{d}
Find the $CH(S') = (v_1,\ldots,v_n)$ in $O(n\log n)$ steps. For each segment $v_iv_{i+1}$, translate it 1 unit to the outside of $CH(S')$ to the direction perpendicular to $v_iv_{i+1}$. Let those line segments be $v'_iv'_{i+1} \subset LCH(S) \subset CH(S)$. $LCH(S)$ can be constructed in $O(n)$ time. Draw arc of 1 unit radius centered at $v_i$, such that it touchs $v'_i$ and $v'_{i+1}$ and doesn't lie inside $CH(S')$. Those arcs be $ACH(S) \subset CH(S)$. $ACH(S)$ can be constructed in $O(n)$ time. $LCH(S)\cup ACH(S) = CH(S)$.\\
The algorithm is in $O(n \log n)$.
\end{document}
\theend
