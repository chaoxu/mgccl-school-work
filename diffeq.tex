\documentclass[letter]{article}
\usepackage{amsmath}
\usepackage{amssymb}
\usepackage{amsfonts}
\usepackage{amsthm}
\usepackage{latexsym}
\usepackage{mathrsfs}
\usepackage{eufrak}
\usepackage[pdftex]{graphicx}
\usepackage{color}
\usepackage{mathrsfs}
\setlength{\topmargin}{-0.5in} \setlength{\textwidth}{6.5in}
\setlength{\oddsidemargin}{0.0in} \setlength{\textheight}{9.1in}

\newlength{\pagewidth}
\setlength{\pagewidth}{6.5in} \pagestyle{empty}

\newcommand{\R}{\mathbb{R}}
\newcommand{\N}{\mathbb{N}}
\newcommand{\B}{\mathfrak{B}}
\newcommand{\Z}{\mathbb{Z}}
\newcommand{\laplace}[1]{\mathscr{L}\{#1\}}
\newcommand{\ilaplace}[1]{\mathscr{L}^{-1}\{#1\}}
\newcommand{\im}{\operatorname{im}}
\newcommand{\fracnoline}[2]{\genfrac{}{}{0pt}{}{#1}{#2}}
\newcommand{\rref}{\operatorname{rref}}
\newcommand{\rank}{\operatorname{rank}}
%\newcommand{\ker}{\operatorname{ker}}
\newtheorem{theorem}{Theorem}[section]
\newtheorem{lemma}[theorem]{Lemma}
\newtheorem{proposition}[theorem]{Proposition}
\newtheorem{corollary}[theorem]{Corollary}

\newenvironment{definition}[1][Definition]{\begin{trivlist}
\item[\hskip \labelsep {\bfseries #1}]}{\end{trivlist}}
\newenvironment{example}[1][Example]{\begin{trivlist}
\item[\hskip \labelsep {\bfseries #1}]}{\end{trivlist}}
\newenvironment{remark}[1][Remark]{\begin{trivlist}
\item[\hskip \labelsep {\bfseries #1}]}{\end{trivlist}}

\begin{document}
\section{First Order DE}
\subsection{Separable Equations}
$\frac{dy}{dx} = H(x,y)$ is separable if $\frac{dy}{dx} = g(x)h(y) = \frac{g(x)}{f(y)}$

$f(y)\frac{dy}{dx} = g(x)$,then integrate both side, $\int{f(y) dy} = \int{g(x) dx}$, then solve for $y$.


\subsection{Linear first order DE}
$\frac{dy}{dx} + P(x) y = Q(x)$
\\
solution:\\
1. find a integrate factor $\mu(x) = e^{\int P(x) dx}$\\
2. multiply both side, reconize left side is the derivative of the other side.\\
3. $\mu(x)y(x) = \int{\mu(x)Q(x)dx} + c$\\
4. solve for $y(x)$\\

\[
y(x) = \frac{\int{\mu(x) Q(x) dx} + c}{\mu(x)}
\]

\newtheorem{thm}{Theorem}
\begin{thm}[Existence and Uniqueness of solutions]
Suppose that both the function $f(x,y)$ and its partial derivative $D_yf(x,y)$ are continuous on some rectangle $R$ in the $xy$-plane that contains the point $(a,b)$ in its interior. Then, for some open interval I containing the point $a$, the initial value problem
\[
\frac{dy}{dx} = f(x,y), y(a) = b
\]
has one and only one solution that is defined on the interval $I$.
\end{thm} 

\begin{thm}[The linear first-order equations]
If the functions $P(x)$ and $Q(x)$ are continous on the open interval $I$ containing the point $x_0$, then the initial value problem
\[
\frac{dy}{dx} + P(x)y = Q(x) , y(x_0) = y_0
\]
has a unique solution $y(x)$ on $I$, given by the formula in with an appropirate value of $C$.
\end{thm}
 
\subsection{Substiution Methods}
$\frac{dy}{dx} = f(x,y)$\\
solution: find a $v = \alpha(x,y)$, such that\\
$y = \beta(x,v)$\\
then use the chain rule\\
\[\frac{dy}{dx} = \frac{\partial \beta}{\partial x}\frac{dx}{dx} + \frac{\partial \beta}{\partial v} \frac{dv}{dx} = \beta_x + \beta_y\frac{dv}{dx}\]

$D_x(x,v)$ and $\beta_v(x,v)$ are known functions.\\
solve for $\frac{dv}{dx} = g(x,v)$\\
we can solve $\frac{dv}{dx} = g(x,v)$ $v = v(x)$ is a solution, then $y = \beta(x, v(x))$ is the solution.\\

\subsubsection{Solveable by subsution}
$\frac{dy}{dx} = F(ax+by+c)$, $v = ax+by+c$\\
$\frac{dy}{dx} = F(\frac{y}{x}$, $v = \frac{y}{x}$ (Homogeneous equations)\\
$\frac{dy}{dx} + P(x)y = Q(x) y^n$, $v = y^{1-n}$ (Bernoulli Equations)\\


\subsection{Exact Differential Equations}
$F(x,y(x)) = C$ or $\frac{\partial F}{\partial x} + \frac{\partial F}{\partial y} \frac{dy}{dx} = 0$ or $M(x,y) + N(x,y)\frac{dy}{dx} = 0$
or $M(x,y) dx + N(x,y)dy = 0$

if there exist $F(x,y)$ such that $\frac{\partial F}{\partial x} =M$ and $\frac{\partial F}{\partial y} =N$. $F(x,y) = C$ is exact.

$dF = F_x dx + F_ydy$ of $F_x(x,y) = M$ and $F_y(x,y)=N$\\
Generally, solving it by setting\\
$F_x(x,y) = M$\\
and we have $F(x,y) = \int{ M dx} + h(y)$\\
Then find $F_y(x,y) = N$, thus help us finding $h(y)$, substute it back and we have $F(x,y)$\\
$\frac{\partial M}{\partial y} = F_{xy} = F_{yx} = \frac{\partial N}{\partial x}$ is necessery condition for $Mdx + Ndy = 0$\\

\subsection{Reducible Second-Order Equations}
Solve $F(x,y',y'') = 0$\\
$p = y'$, $y'' = \frac{dp}{dx}$\\
$F(x,p,p')=0$\\
$y(x) = \int{p(x,C_1) dx + C_2}$\\

Solve $F(y,y',y'') =0$\\
$p=y'$, $y''=p\frac{dp}{dy}$\\

$F(y,p,p\frac{dp}{dy}) = 0$\\
Find $p(y,C_1)$\\
then $x(y) = \int{\frac{dy}{p(y,C_1)} + C_2} = \int{\frac{1}{p}dy}$\\

\section{Higher Order Linear Equations}
\subsection{Homogeneous Second-Order Linear Equations}
$y'' + B(x)y'+C(x)y = 0$\\

\subsubsection{Principle of superposition for Homogeneous equations}
Let $y_1$ and $y_2$ be two solutions of the homogeneous linear equation on the interval $I$. If $c_1$ and $c_2$ are constants, then the linear combination $y=c_1y_1 + c_2y_2$ is also a solution of the equation on $I$.

\begin{thm}[Existence and uniqueness of Linear Equations]
Suppose that the function $p$, $q$ and $f$ are continous on the open interval $I$ containing the point $a$. Then, given any two number $b_0$ and $b_1$, the equation
\[
y''+p(x)y' + q(x)y = f(x)
\]
has a unique (that is, one and only one) solution on the entire interval $I$ that satisfies the initial conditions.
\[
y(a) = b_0, y'(a) =b_1
\]
\end{thm}

\begin{thm}[Wronskain of Solutions]
Suppose that $y_1$ and $y_2$ are two solutions of the homogeneous second-order linear equation.
\[
y'' + p(x)y' + q(x)y = 0
\]
on an open interval $I$ on which $p$ and $q$ are continous.
(a) If $y_1$ and $y_2$ are linearly dependent, then $W(y_1,y_2) \equiv 0$ on $I$.
(b) If $y_1$ and $y_2$ are linearly independent, then $W(y_1, y_2) \neq 0$ at each point of $I$.
\end{thm}


\begin{thm}[General solutions of Homogeneous Equations]
Let $y_1$ and $y_2$ be two linearly independent solutions of the homogeneous equation
\[
y'' + p(x)y' + 	q(x)y=0
\]
with $p$ and $q$ continuous on the open interval $I$. If $Y$ is any solution whatsoever on $I$, then there exist number $c_1$ and $c_2$ such that
\[
Y(x) = c_1y_1(x) + c_2y_2(x)
\]
for all $x$ in $I$
\end{thm}

\begin{thm}[Abel's Theorem]
If $y_1$ and $y_2$ are solutions of the differential equation
\[
y''+p(x)y'+q(x)y = 0
\]
where $p$ and $q$ are continous on an open interval $I$, then the Wronskian $W(y_1,y_2)(x)$ is given by
\[
W(y_1,y_2)(x) = c e^{-\int{p(x)dx}}
\]
where c is a certain constant that depends on $y_1$ and $y_2$, but not on $x$.
\end{thm}

\subsection{Linear Second-Order Equations with Constant Coefficients}
\[
ay''+by'+cy=0
\]
We look for solutions $r_1$ and $r_2$ to $ar^2+br+c=0$ ,then $y(x) = c_1e^{r_1x} + c_2e^{r_2x}$ is the general solution.

If there are repeated roots, then $y(x) = (c_1+c_2x)e^{r_1x}$

\subsection{General Solution of nth-Order Linear Equations}
$y^{(n)} + p_1(x)y^{(n-1)} + \cdots + p_{n-1}(x)y' + p_n(x)y = 0$ is the homogeneous linear equation associated with the following equation $y^{(n)} + p_1(x)y^{(n-1)} + \cdots + p_{n-1}(x)y' + p_n(x)y = f(x)$
Let $y_1,y_2,\ldots,y_n$ be $n$ solutions of the homogeneous linear equation of order $n$ on the interval $I$. If $c_1, c_2, \ldots, c_n$ are constants, then the linear combination
\[
y = c_1y_1 + c_2y_2 + \cdots + c_ny_n
\]
is also a solution on $I$

General solutions of Homogeneous Equations are similar to the one for 2nd order DE.
\subsection{Initial value problem}
Suppose that the function $p_1, p_2, \ldots, p_n$, and $f$ are continous on the open interval $I$ containing the point $a$, Then, given $n$ numbers $b_0, b_1, \ldots, b_n$, the $n$th-order linear equation
 $y^{(n)} + p_1(x)y^{(n-1)} + \cdots + p_{n-1}(x)y' + p_n(x)y = f(x)$
has a unique solution on the entire interval $I$ that satisfies the $n$ initial condisions.
$y(a) = b_0, y'(a) = b_1, \ldots, y^{(n-1)}(a) = b_{n-1}$

\subsection{Linear Independent}
If $c_1f_1+c_2f_2 + \cdots + c_nf_n = 0$ on the interval $I$, then they are linearly dependent on interval $I$.

The Wronskian of the $n$ function $= 0$, then they are linearly dependent. It is a generalization of the Wronskain of solution for 2nd order differential equations.

\subsection{Reduction of Order}
$y''+p(x)y'+q(x)y=0$, and $y_1$ is a solution, then the other solution $y_2 = v(x)y_1(x)$. Where $y_1v'' + (2y_1'+py_1)v' = 0$. Thus it is reduced to first order.

\subsection{Homogeneous Equations with Constant Coefficients}
$a_ny^{(n)} + a_{n-1}y^{(n-1)} + \cdots + a_2y'' + a_1y'+a_0y = 0$ can be solved similarly as the 2nd order version.
For distinct real roots, the result is
$y(x) = c_1e^{r_1x}+c_2e^{r_2x} + \cdots + c_ne^{r_nx}$

\paragraph{With repeated roots}

If it have root $r$ of multiplicity $k$, then the part of the general solution for $r$ is the form
\[
(c_1+c_2x+c_3x^2 + \cdots + c_kx^{k-1})e^{rx}
\]

\paragraph{Complex roots}
If there is a unrepeated pair of complex conjugate roots $a \pm bi$, then the corresponding part of a general solution has the form
\[
e^{ax}(c_1\cos(bx)+c_2\sin(bx))
\]

\paragraph{Repeated Complex Roots}
if $a \pm bi$ has multiplicity $k$, then the part of the general solution is 
\[
\sum_{p=0}^{k-1} x^p e^{ax}(c_i \cos(bx) + d_i \sin(bx))
\]


\subsection{Nonhomogeneous Equations}
\[
y^{(n)} + p_1(x)y^{(n-1)} + \cdots + p_{n-1}(x)y' + p_n(x)y = f(x)
\]

Suppose there is a solution for the above for a specific solution, $y_p$. $Y$ is any other solution. $Y = y_c+y_p$, where $y_c$ is the solution of the associated homogeneous equation. $y_c$ is a complementary function of the nonhomogeneous equation.


\subsubsection{Solution of Nonhomogeneous Equations}

Let $y_p$ be a particular solution of the nonhomogeneous equation on an open interval $I$ where the function $p_i$ and $f$ are continous. Let $y_1, y_2, \ldots, y_n$ be linearly independent solutions of the associated homogeneous equation. If $Y$ is any solution whatsoever on $I$ for the nonhomogeneous equation, then there exist numbers $c_1, c_2, \ldots, c_n$ such that
\[
Y(x) = c_1y_1(x) + c_2y_2(x) + \cdots + c_ny_n(x) + y_p(x)
\]
for all $x$ in $I$

\subsubsection{The method of undetermined coefficients}


\paragraph{Rule 1}
Suppose that no term appearing either in $f(x)$ or in any of its derivatives satisfies the associated homogeneous equation $Ly=0$. Then take as a trial solution for $y_p$ a linear combination of all linearly independent such terms and their derivativees. Then determine the coefficients by substitution of this trial solution into the nonhomogeneous equation $Ly = f(x)$

\paragraph{Rule 2}
If the function $f(x)$ is of either form $P_m(x)e^{rx}\cos(kx)$ or $P_m(x)e^{rx}\sin(kx)$, take as the trial solution\\
\[
y_p(x) = x^s\left[(\sum_{i=0}^m{A_ix^i})e^{rx}\cos(kx)+(\sum_{i=0}^m{B_ix^i})e^{rx}\sin(kx)\right]
\]
where $s$ is the smallest nonnegative integer such that such that no term in $y_p$ duplicates a term in the complementary function $y_c$. Then determine the coefficients by substituting $y_p$ into the nonhomogeneous equation.

\subsubsection{Variation of Parameters}
Provided we know the general solution $y_c = c_1y_1 + c_2y_2 + \cdots + c_ny_n$ for a associated homogeneous equation, then we can use variation of parameters to find $y_p$.
\[
y_p(x) = u_1(x)y_1(x) + u_2(x)y_2(x) + \cdots + u_n(x)y_n(x)
\]
Find such $u$'s. It is possible for all order $n\geq 2$.

\paragraph{For order of 2}
 If the nonhomogeneous equation $y''+P(x)y'+Q(x)y = f(x)$ has complementary function $y_c(x) = c_1y_1(x) + c_2y_2(x)$, then a particular solution is given by

\[
y_p(x) = -y_1(x) \int{\frac{y_2(x)f(x)}{W(x)}dx} + y_2(x) \int{\frac{y_1(x)f(x)}{W(x)}dx}
\]

where $W = W(y1,y2)$ is the Wronskian of the two independent solutions $y_1$ and $y_2$ of the associated homogeneous equation.

\[u'_1y_1 + u'_2y_2 = 0\]
\[u'_1y'_1+u'_2y'_2=f(x)\]
find $u_1$, $u_2$ and plug in to get the particular solution.

\section{Series solutions of second order linear equations}

\subsection{Review of Power Series}
\begin{definition}
A power series $\sum_{n=0}^{\infty} a_n(x-x_0)^n$ is said to \textbf{converge} at a point $x$ if
\[
\lim_{m\to \infty} \sum_{n=0}^{m} a_n(x-x_0)^n
\]
exists. The series converges for $x=x_0$; it may converge for all $x$ or some $x$.
\end{definition}

\begin{definition}
A power series $\sum_{n=0}^{\infty} a_n(x-x_0)^n$ is said to \textbf{converge absolutely} at a point $x$ if
\[
\sum_{n=0}^{\infty} |a_n(x-x_0)^n|
\]
exists. If the series converges absolutely, then the series converge.
\end{definition}

\begin{theorem}
The most useful test for the absolute convergence of a power series is the ratio test. If $a_n\neq 0$, and if for a fixed value of $x$
\[
\lim_{n\to \infty} |\frac{a_{n+1}(x-x_0)^{n+1}}{a_n(x-x_0)^n}| = |x-x_0| \lim_{n\to \infty} |\frac{a_{n+1}}{a_n}| = l
\]
then the power series converges absolutely at that value of $x$ if $l < 1$, and diverges if $l>1$. If $l=1$ the test is inconclusive.
\end{theorem}

\begin{theorem}
If the power series  $\sum_{n=0}^{\infty} a_n(x-x_0)^n$ converges at $x = x_1$, it converges absolutely for $|x-x_0| < |x_1 - x_0|$; and if it diverges at $x=x_1$, it diverges for $|x-x_0| > |x_1-x_0$.
\end{theorem}

\begin{definition}
There is a nonnegative number $\rho$, called the \textbf{radius of convergence}, such that  $\sum_{n=0}^{\infty} a_n(x-x_0)^n$ converges absolutely for $|x-x_0|<\rho$ and diverges for $|x-x_0|>\rho$. For a series that converges nowhere except at $x_0$, we define $\rho$ to be zero; for a series that converges for all $x$, we say that $\rho$ is infinite. If $\rho>0$, then the interval $|x-x_0<\rho|$ is called the \textbf{interval of convergence}. The series may either converge or diverge when $|x-x_0| = \rho$.
\end{definition}

If  $\sum_{n=0}^{\infty} a_n(x-x_0)^n$ and  $\sum_{n=0}^{\infty} b_n(x-x_0)^n$ converge to $f(x)$ and $g(x)$, respectively, for $|x-x_0|<\rho$, $\rho>0$, then the following are true for $|x-x_0|<\rho$.
\[
f(x) \pm g(x) = \sum_{n=0}^{\infty} (a_n\pm b_n)(x-x_0)^n
\]
\[
f(x)g(x) = \sum_{n=0}^{\infty} (c_n)(x-x_0)^n
\]
where $c_n = \sum_{i=0}^n a_i b^{n-i}$
\[
\frac{f(x)}{g(x)} = \sum_{n=0}^{\infty} (d_n)(x-x_0)^n
\]
which can be done by using below.
\[
\sum_{n=0}^{\infty} (a_n)(x-x_0)^n = \sum_{n=0}^{\infty} (d_n)(x-x_0)^n\sum_{n=0}^{\infty} (b_n)(x-x_0)^n
\]
In case of division, the radius of convergence of the resulting power series may be less than $\rho$

\begin{theorem}
$f$ is continuous and has derivatives of all orders of $|x-x_0|<\rho$. Further, $f',f'',\ldots$ can be computed by differentiating the series term wise; that is,
\[f'(x) = \sum_{n=1}^\infty n a_n (x-x_0)^{n-1}\]
\[f''(x) = \sum_{n=2}^\infty n(n-1) a_n (x-x_0)^{n-2}\]
and so forth, and each of the series converges absolutely for $|x-x_0|<\rho$.
\end{theorem}

\begin{definition}
\[ a_n = \frac{f^{(n)}(x_0)}{n!} \]
The series is called the Taylor series for $f$ about $x=x_0$.
\end{definition}

\begin{theorem}
If $\sum_{n=0}^\infty a_n(x-x_0)^n = \sum_{n=0}^\infty b_n(x-x_0)^n$ for each $x$, then $a_n = b_n$. If $\sum_{n=0}^\infty a_n(x-x_0)^n = 0$ for each $x$, then $a_n = 0$.
\end{theorem}

\begin{definition}
\[
f(x) = \sum_{n=0}^\infty \frac{f^{(n)}(x_0)}{n!}(x-x_0)^n
\]
with a radius of convergence $\rho > 0$ is said to be analytic at $x=x_0$.
\end{definition}

\begin{definition}
\[
P(x)y''+Q(x)y'+R(x)y = 0
\]
Then $x_0$ is a \textbf{ordinary point} if $P(x)\neq 0$. It is a \textbf{regular singular point} if
$(x - x_0) \frac{Q(x)}{P(x)}$ and $(x-x_0)^2 \frac{R(x)}{P(x)}$ are analytic at $x = x_0$. i.e. radius of convergence $>0$

\subsection{Series solutions near an ordinary point}
\begin{theorem}
If $x_0$ is an ordinary point of the differential equation
\[
P(x)y''+Q(x)y'+R(x)y=0
\]
the general solution is
\[
y = \sum_{n=0}^\infty a_n(x-x_0)^n = a_0 y_1(x) + a_1 y_2(x)
\]
The radius of convergence of $y_1$ and $y_2$ is at least as large as the radii of convergence of $Q/P$ and $R/P$.
\end{theorem}

\subsection{Euler Equations}
\begin{theorem}
General solution of the Euler equation
\[
x^2 y'' + \alpha xy' + \beta y=0
\]
in any interval not containing the origin are determined by the roots $r_1$ and $r_2$ of the equation
\[
F(r)=r(r-1)+\alpha r + \beta = 0
\]
if the roots are real and different, we have
\[
y = c_1 |x|^{r_1} + c_2 |x|^{r_2}
\]
same
\[
y = (c_1+c_2 \ln|x|)|x|^{r}
\]
complex roots $a\pm bi$
\[
y = |x|^a (c_1 \cos(b \ln|x|) + c_2 \sin(b \ln |x|))
\]
\end{theorem}

\subsection{Series solutions near a regular singular point}
First reduce it to Euler equation.
\[
P(x)y''+Q(x)y'+R(x)y = 0
\]
let $p(x) = \frac{Q(x)}{P(x)}$ and $q(x) = \frac{R(x)}{P(x)}$, then we have
\[
x^2y''+x[xp(x)]y'+[x^2q(x)]y=0
\]
Let $p_0 = \lim_{x\to 0} \frac{xQ(x)}{P(x)}$ and $q_0 = \lim_{x\to 0} \frac{x^2R(x)}{P(x)}$.
we now have
\[
x^2y''+p_0xy'+q_0y=0
\]
\begin{theorem}
Find the Euler equation's roots.
Let
\[
y = \sum_{n=0}^{\infty} a_n x^{n+r}
\]
Substitute into the original formula.\\
Find the recurrence relation to determine the coefficient.\\


\end{theorem}
\begin{theorem}
\[ x^2y''+x[xp(x)]y'+[x^2q(x)]y=0 \]
where $x=0$ is a regular singular point. Then the functions $xp(x)$ and $x^2q(x)$ are analytic at $x=0$.
\[
xp(x) = \sum_{n=0}^\infty p_n x^n
\]
\[
x^2 q(x) = \sum_{n=0}^\infty q_n x^n
\]
Let $r_1$ and $r_2$ be the roots of the \textbf{indicial equation}
\[
F(r) = r(r-1) + p_0 r + q_0 = 0
\]
the roots are called \textbf{exponents of the singularity}
where $r_1 \geq r_2$. Then solution of the form
\[
y_1(x) = |x|^{r_1} \left( 1+\sum_{n=1}^\infty a_n(r_1)x^n \right)
\]
If $r_1 = r_2$, the second solution is
\[
y_2(x) = y_1(x) \ln|x| + |x|^{r_1} \sum_{n=1}^\infty b_n(r_1)x^n
\]
If $r_1-r_2 = N$
\[
y_2(x) = a y_1(x) \ln|x| + |x|^{r_2}  \left( 1+\sum_{n=1}^\infty c_n(r_2)x^n \right)
\]

The coefficients can be found with the recurrence relation\\
\[
F(r+n)a_n + \sum_{k=0}^{n-1} a_k[(r+k)p_{n-k} + q_{n-k}] = 0
\]
or substitute $y$ back.
\end{theorem}
\end{definition}
\section{Definition of the Laplace Transform}
\begin{definition}
A \textbf{integral transform} is a relation of the form 
\[
F(s) = \int_\alpha^\beta K(s,t) f(t) dt
\]
$K$ is the \textbf{kernel} of the transformation.
\end{definition}

\begin{definition}
The \textbf{Laplace transform} of $f$, denoted $\laplace{f(t)}$, is defined by the equation
\[
\laplace{f(t)} = F(s) = \int_0^\infty e^{-st} f(t) dt
\]
\end{definition}

\begin{theorem}
If $f$ is a piecewise continuous on the interval $0\leq t \leq A$ for any positive $A$, and $|f(t)| \leq Ke^{at}$ when $t\geq M$. $K,M\geq 0$. Then the $\laplace{f(t)}  = F(s)$ exists for $s > a$.
\end{theorem}

\begin{theorem}
Laplace transform and it's inverse is a linear transform.
\end{theorem}

\subsection{Solution of Initial Value Problems}
\begin{theorem}
Suppose that $f$ is continuous and that $f'$ is piecewise continuous on any interval $0\leq t \leq A$. Suppose further that there exist constants $K, a$, and $M$ such that $|f(t)| \leq Ke^{at}$ for $t\geq M$. Then $\laplace{f'(t)}$ exists for $s>a$, and
\[
\laplace{f'(t)} = s \laplace{f(t)} - f(0)
\]
\end{theorem}

\begin{corollary}
Suppose that the function $f, f', \ldots, f^{(n-1)}$ are continuous, and that $f^{(n)}$ is piecewise continuous on any interval $0\leq t \leq A$.  Suppose further that there exist constants $K, a$, and $M$ such that $|f(t)| \leq Ke^{at},|f'(t)| \leq Ke^{at},\ldots,|f^{(n-1)}(t)| \leq Ke^{at}$ for $t\geq M$. Then $\laplace{f^{(n)}(t)}$ exists for $s>a$, and
\[
\laplace{f^{(n)}(t)} = s^n \laplace{f(t)} - s^{n-1} f(0) - \ldots -  s f^{(n-2)}(0) - f^{(n-1)}(0)
\]
\end{corollary}

\subsection{Step Functions}
\begin{definition}
\textbf{unit step function} denoted by $u_c$ is defined by 
\[
u_c(t) = \left\{
\fracnoline{0, \indent t<c}
{1, \indent t\geq c} \right. \indent c\geq 0
\]
\end{definition}

\begin{theorem}[Laplace transform of a unit step function]
\[\laplace{u_c(t)} = \frac{e^{-cs}}{s},\indent s>0\]
\end{theorem}

\begin{theorem}
If $F(s) = \laplace{f(t)}$ exists for $s>a\geq 0$, and if $c$ is a positive constant, then
\[
\laplace{u_c(t) f(t-c)} = e^{-cs}\laplace{f(t)} = e^{-cs}F(s), \indent s>a
\]
Conversely, if $f(t) = \ilaplace{F(s)}$, then
\[
u_c(t)f(t-c) =  \ilaplace{e^{-cs}F(s)}
\]
\end{theorem}

\begin{theorem}
If $F(s) = \laplace{f(t)}$ exists for $s>a\geq 0$, and if $c$ is a constant, then
\[
\laplace{e^{ct} f(t)} =F(s-c), \indent s>a
\]
Conversely, if $f(t) = \ilaplace{F(s)}$, then
\[
e^{ct} f(t) =  \ilaplace{F(s-c)}
\]
\end{theorem}

\subsection{Differential Equations with Discontinuous Forcing Functions}
Just some examples.

\subsection{Impulse Functions}
\begin{definition}
\[  g(t) = ay'' + by'+cy \]
g(t) is large during a short interval $t_0 - \tau < t < t_0 + \tau$ and is otherwise zero.
$I(\tau)$, defined by
\[
I(\tau) = \int_{t_0-\tau}^{t_0+\tau} g(t) dt = \int_{-\infty}^\infty g(t) dt \]
is the total \textbf{impulse} of the force $g(t)$ over the time interval $(t_0-\tau, t_0+\tau)$
\end{definition}

\begin{definition}
A \textbf{unit impulse function} $\delta$ defined to have the property
\[\delta(t) = 0,\indent t\neq 0\]
\[\int_{-\infty}^{\infty} \delta(t) dt = 1\]
\end{definition}

\begin{definition}
\[\laplace{\delta(t-t_0)} = \lim_{\tau\to 0} \laplace{d_\tau (t-t_0)} \]
\end{definition}

\begin{theorem}
\[\laplace{d_\tau (t-t_0)} = \int_{t_0-\tau}^{t_0+\tau} e^{-st}d_\tau(t-t_0) dt \]
\[ = \frac{1}{2\tau s} e^{-st_0} (e^{st} - e ^{-st}) \]
\[=\frac{\sinh s\tau}{s\tau} e^{-st_0}\]
\end{theorem}

\begin{definition}
\[\int_{-\infty}^{\infty} \delta(t-t_0) f(t) dt = \lim_{\tau\to 0} \int_{-\infty}^{\infty} d_\tau(t-t_0) f(t) dt\]
\end{definition}

\begin{theorem}
\[ \int_{-\infty}^{\infty} d_\tau(t-t_0) f(t) dt = \frac{1}{2\tau}\int_{t_0-\tau}^{t_0+\tau} f(t) dt = f(t^*)\]
Where $t_0-\tau<t^*<t_0+\tau$
\end{theorem}

\subsection{The Convolution Integral}
\begin{theorem}
If $F(s) = \laplace{f(t)}$ and $G(s) = \laplace{g(t)}$ both exist for $s>a\geq 0$, then
\[H(s) = F(s)G(s) = \laplace{h(t)}, \indent s>a \]
where
\[ h(t) = \int_0^t f(t-\tau)g(\tau) d\tau = \int_0^t f(\tau)g(t-\tau) d\tau \] 
The function $h$ is known as the \textbf{convolution} of $f$ and $g$; the integrals are known as \textbf{convolution integrals}.
\end{theorem}

\begin{theorem}
\[(f * g)(t) = \int_0^t f(t-\tau)g(\tau) d\tau\]
Then
\[f*g = g*f\]
\[f*(g_1+g_2) = f*g_1 + f*g_2\]
\[(f*g)*h = f*(g*h)\]
\[f*0 = 0*f = 0\]
\end{theorem}

\section{Partial differential equations}
\subsection{Two-point boundary value problems}
\begin{definition}
For the following DE
\[ y'' + \lambda y = 0 \]
$\lambda$ is the \textbf{eigenvalue}, nontrivial solutions are \textbf{eigenfunctions}. 
\end{definition}

\begin{theorem}
For boundary values $y(0) = 0$, $y(L)=0$
\[ y'' + \lambda y = 0 \]
If $\lambda > 0$
we have 
\[ y'' + u^2 y = 0 \]
\[y = c_1 \cos ux + c_2 \sin ux\]

If $\lambda < 0$
\[ y'' - u^2 y = 0 \]
\[y = c_1 \cosh ux + c_2 \sinh ux\]

If $\lambda = 0$
\[ y''= 0 \]
\[ y = c_1 x + c_2 \]

The eigenvalues and the eigenfunctions are
\[
\lambda_n = n^2 \pi^2 /L^2
\]

\[
y_n(x) = \sin(n\pi x /L)
\]
\end{theorem}
\subsection{The Fourier convergence theorem}
\begin{theorem}
$f$ and $f'$ are piecewise continuous on $-L\leq x < L$
\[
f(x) = \frac{a_0}{2} + \sum_{m=1}^\infty (a_m \cos \frac{m\pi x}{L} + b_m \sin \frac{m\pi x}{L})
\]
coefficients are
\[
a_m = \frac{1}{L} \int_{-L}^L f(x) \cos \frac{m\pi x}{L} dx
\]
\[
b_m = \frac{1}{L} \int_{-L}^L f(x) \sin \frac{m\pi x}{L} dx
\]
Converge to (f(x+)+f(x-))/2 at points where $f$ is discontinuous.
\end{theorem}

\subsection{Even and odd functions}
\begin{theorem}
\begin{enumerate}
\item Sum and product of even functions are even
\item sum of two odd functions are odd, product are even
\item sum of odd and even is neither even nor odd, the product is odd
\item $f$ is even, then 
\[
\int_{-L}^L f(x) dx = 2 \int_0^L f(x) dx
\]
\item $f$ is odd, then 
\[
\int_{-L}^L f(x) dx = 0
\]
\end{enumerate}
\end{theorem}

\begin{theorem}
For even functions,
\[f(x) = \frac{a_0}{2} + \sum_{n=1}^\infty a_n \cos \frac{n\pi x}{L}\]
\[a_n = \frac{2}{L} \int_{0}^L f(x) \cos \frac{n\pi x}{L} dx \]
For odd functions, we have

\[f(x) =+ \sum_{n=1}^\infty b_n \sin \frac{n\pi x}{L}\]
\[b_n = \frac{2}{L} \int_{0}^L  f(x) \sin \frac{n\pi x}{L} dx \]
\end{theorem}

\subsection{Heat equation}
\begin{definition}
The following is the \textbf{heat conduct equation}
\[
\alpha^2 u_{xx} = u_{t}
\]
\end{definition}

\begin{theorem}
The steady temperature solution, given  $u(0,t) = T_1$, $u(L,t) = T_2$, is
\[
v(x) = (T_2 - T_1)\frac{x}{L} + T_1
\]
\end{theorem}
\begin{theorem}
Given  $u(0,t) = T_1$, $u(L,t) = T_2$, $u(x,0) = f(x)$
we have
\[
u(x,t) = (T_2-T_1)\frac{x}{L} + T_1 + \sum_{n=1}^\infty c_n e^{-n^2\pi^2\alpha^2 t/L^2} \sin \frac{n\pi x}{L}
\]
where
\[
c_n = \frac{2}{L} \int_0^L [f(x) - (T_2-T_1)\frac{x}{L} - T_1] \sin \frac{n\pi x}{L}
\]
\end{theorem}

\subsection{Wave equation}
\begin{definition}
This is the \textbf{wave equation}
\[
a^2 u_{xx} = u_{tt}
\]
\end{definition}

\begin{theorem}[Elastic string with nonzero initial displacement]
Boundary condition: $u(0,t) = u(L,t) = 0$, initial conditions $u(x,0) = f(x)$, $u_t(x,0) = 0$.
Then
\[
u(x,t) = \sum_{n=1}^\infty c_n u_n (x,t) = \sum_{n=1}^\infty c_n \sin \frac{n\pi x}{L} \cos \frac{n\pi at}{L}
\]
where
\[
c_n = \frac{2}{L}\int_0^L f(x) \sin \frac{n\pi x}{L} dx
\]
\end{theorem}

\begin{theorem}[General problem for the elastic string]
Boundary condition: $u(0,t) = u(L,t) = 0$, initial conditions $u(x,0) = 0$, $u_t(x,0) = g(x)$.
then
\[
u(x,t) = \sum_{n=1}^\infty k_n u_n (x,t) = \sum_{n=1}^\infty k_n \sin \frac{n\pi x}{L} \sin \frac{n\pi at}{L}
\]
where
\[
k_n = \frac{2}{n\pi a}\int_0^L g(x) \sin \frac{n\pi x}{L} dx
\]
\end{theorem}
\end{document}

