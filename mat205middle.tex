\documentclass[letter]{article}
\usepackage{amsmath}
\usepackage{amssymb}
\usepackage{amsfonts}
\usepackage{amsthm}
\usepackage{latexsym}
\usepackage[pdftex]{graphicx}
\usepackage{color}

\setlength{\topmargin}{-0.5in} \setlength{\textwidth}{6.5in}
\setlength{\oddsidemargin}{0.0in} \setlength{\textheight}{9.1in}

\newlength{\pagewidth}
\setlength{\pagewidth}{6.5in} \pagestyle{empty}

\newcommand{\R}{\mathbb{R}}
\newcommand{\N}{\mathbb{N}}
\newcommand{\Z}{\mathbb{Z}}
\newcommand{\curl}{\operatorname{curl}}

\newtheorem{theorem}{Theorem}[section]
\newtheorem{lemma}[theorem]{Lemma}
\newtheorem{proposition}[theorem]{Proposition}
\newtheorem{corollary}[theorem]{Corollary}

\newenvironment{definition}[1][Definition]{\begin{trivlist}
\item[\hskip \labelsep {\bfseries #1}]}{\end{trivlist}}
\newenvironment{example}[1][Example]{\begin{trivlist}
\item[\hskip \labelsep {\bfseries #1}]}{\end{trivlist}}
\newenvironment{remark}[1][Remark]{\begin{trivlist}
\item[\hskip \labelsep {\bfseries #1}]}{\end{trivlist}}

\title{Evil Notes}
\date{}

\begin{document}
\maketitle
\vspace{-.5in}

\begin{theorem}
$f(x)$ is differentiable if
\[
\lim_{x\to x_0} \frac{f(x) - (f(x_0) + M(x-x_0))}{|x-x_0|} = 0
\]
Where M is the Jocabian Matrix.
\end{theorem}

\begin{theorem}
Affine approximation a function f(x) are in the following form
\[
A(x) = M(x) + y
\]
Where $M(x)$ is the jacobian matrix.\\
A way fo find it when $f$ is known.
\[ A(x) = f(x_0) + f'(x_0)(x-x_0) \]
\end{theorem}

\begin{definition}
The directional derivative is defined as
\[
\lim_{h\to 0} \frac{f(a+h\vec v) - f(a)}{h} = f'(a)\vec v
\]
also written as
\[
\frac{\partial f}{\partial u}(x) = f'(x) u
\]
where $u$ is a unit vector.\\
\end{definition}

\begin{theorem}
To find $df$ at point $x$ on vector $\vec v$, we have
\[
f'(x) \vec v
\]
To find $df$ at point $x$, which the direction of $\vec v$, we have 
\[
f'(x) \frac{\vec v}{|\vec v|}
\]
\end{theorem}
\begin{definition}[The gradient]
\[
\nabla f(x) = (\frac{\partial f}{\partial x_1}(x) \ldots \frac{\partial f}{\partial x_n}(x))
\]
\end{definition}
\begin{theorem}
\[
(g(f(t)))' = \nabla g(f(t)) \cdot f'(t)
\]
It's a special case of the Chain Rule.
\end{theorem}
\begin{theorem}
$\nabla f(x)$ is the direction where $f(x)$ increase most rapidly. $|\nabla f(x)|$ is the magnitude of the maxim increase.
\end{theorem}
\begin{theorem}
Normal vector to $S$ at $x_0$ is $\nabla f(x_0)$.\\
Tangent space to $S$ at $x_0$ is all $x$ such that
\[
\nabla f(x_0) \cdot(x-x_0) = 0
\]
\end{theorem}

\begin{theorem}
Jacobian matrix for polar and spherical
\[
\left( \begin{array}{cc}
\cos(\theta) & -r\sin(\theta) \\
\sin(\theta) & r\cos(\theta) \\
\end{array} \right)
\]

\[ \left( \begin{array}{ccc}
\sin \phi \cos \theta & r \cos \phi \cos \theta & -r\sin \phi \sin \theta \\
\sin \phi \sin \theta & r\cos \phi \sin \ \theta & r\sin \phi \cos \theta \\
\cos \theta & -r\sin \phi & 0 \end{array} \right)\] 

\[
\left( \begin{array}{c}
x\\
y\\
\end{array} \right) = 
\left( \begin{array}{c}
r \cos \theta\\
r \sin \theta\\ \end{array} \right)
\]

The G's
\[
\left( \begin{array}{cc}
1&0\\
0&r^2\\
\end{array} \right)
\]

\[
\left( \begin{array}{ccc}
1&0&0\\
0&r^2&0\\
0&0&r^2 \sin^2\phi\\
\end{array} \right)
\]
\end{theorem}

\begin{theorem}
\[\int_\gamma \nabla f \cdot dx = f(b) - f(a)\]
Where $\gamma$ is a piecewise smooth curve that start and end at point $a$ and $b$. The path doesn't matter, thus we also have.
\[\int_a^b \nabla f dx = \int_\gamma \nabla f \cdot dx\]
\end{theorem}

\begin{theorem}[Chain rule]
\[(g(f(x)))' = g'(f(x))f '(x)\]
\end{theorem}

\begin{theorem}[Inverse Function Theorem]
\[(f^{-1}(f(x_0)))' = (f'(x_0))^{-1}\]
Inverse function exist locally in the neighborhood of $x_0$, if $f'(x_0)$ has a inverse.
\end{theorem}

\begin{theorem}[Implicit function theorem]
If $F:\R^{n+m}\to \R^m$ for some $x_0$ in $R^n$, $y_0$ in $R^m$.
\begin{enumerate}
\item $F(x_0,y_0) = 0$
\item $F_y(x_0,y_0)$ has an inverse
\end{enumerate}
Then $f: \R^n\to \R^m$ is defined implicitly on neighborhood $N$ of $x_0$, so $f(x_0) = y_0$, $F(x,f(x)) = 0$ for all $x$ in $N$.
\end{theorem}

\begin{theorem}[implicitly defined level set affine approximation]
$F:\R^{n+m}\to \R^m$. If we have level set $F(x) = z_0$, $F$ differentiable at $x_0$, then
\[A(x) = F(x_0) + F(x_0)(x-x_0)\]
\[A(x) = z_0\]
lead to
\[ F'(x_0)(x-x_0) = 0\]
Which is the tangent plane.
\end{theorem}

\begin{theorem}
Tangent to level set $F(x) = z_0$ at $x_0$ exists, if and only is $F'(x_0)$ have $m$ linearly independent columns.
\end{theorem}

\begin{theorem}[Lagrange multiplier method]
$G:\R^n\to \R^m$,$n>m$. Let coordinate function $G_i$($1\leq i\leq m$) such that $G_i(x_1,\ldots,x_n)=0$ at point $x_0$. $G'(x_0)$ has $m$ columns linearly independent.\\
$x_0$ is a extreme value of $f:\R^n \to \R$ when restricted to $S$, then $x_0$ is critical point of 
\[f+ \lambda_1 G_1 + \ldots + \lambda_m G_m\]
or
\[f'(x_0)+ \lambda_1 G_1'(x_0) + \ldots + \lambda_m G_m'(x_0) = 0\]
\end{theorem}

\begin{theorem}[General change of variable formula]
For transformation $T$,
\[
\int_{T(R)} f dV = \int_{R} (f \circ T) |det T'| dV
\]
\end{theorem}

\begin{theorem}[Line integral]
\[
\int_a^b F(g(t)) g'(t) dt
\]
Is the line integral of field $F$ over curve $g(t)$ is the curve.
\end{theorem}

\begin{definition}
$\omega^p$ is the $p$-form.
\end{definition}

\begin{theorem}
image $S = \partial R$ in $\R^n$.
\[\int_S \omega^p - \int_R \omega^p_g (\frac{\partial g}{\partial u_1}, \ldots, \frac{\partial g}{\partial u_p}) dV\]
\end{theorem}

\begin{definition}[Exterior derivatives]
The exterior derivative $d$ of a $k$-form $\omega$, denoted $d\omega$, takes a $k+1$-parallelogram and returns a number, as follows:
\[d\omega(P_x^0(v_1,\ldots,v_{k+1})) = \lim_{h\to 0}\frac{1}{h^{k+1}} \int_{\partial P_x^0(hv_1,\ldots,hv_k+1)}\omega\]
\end{definition}

\begin{theorem}[Computing the exterior derivative of a $k$-form]
\[\omega = \sum_{1\leq i_1<\ldots< i_k\leq n} a_{i_1,\ldots,i_k} dx_{i_1}\wedge \ldots \wedge dx_{i_k} \]
\[
d\omega = \sum_{1\leq i_1<\ldots< i_k\leq n} (da_{i_1,\ldots,i_k}) dx_{i_1}\wedge \ldots \wedge dx_{i_k} 
\]
\end{theorem}

\begin{theorem}[Green's Theorem]
Let $\curl F(x) = \frac{\partial F_2}{\partial x_1} - \frac{\partial F_1}{\partial x_2}$
\[
\int_D(\frac{\partial F_2}{\partial x_1} - \frac{\partial F_1}{\partial x_2}) dx_1 dx_2 = \int_D \curl F dA = \int_\gamma F_1dx_1+F_2 dx_2
\]
$\gamma$ is the curve that bounds the surface $D$.

\end{theorem}

\begin{definition}
A vector field $F$ for which there is a real-valued function $f$ such that $F = \nabla f$ is called a gradient field. In that case $f$ is called the potential of $F$.
\end{definition}

\begin{theorem}
Let $F$ be a continuous vector field defined in a polygonally connected open set $D$ of $\R^n$, then
\begin{enumerate}
\item $F$ is the gradient of a function, $f$, continuously differentiable in $D$.
\item The line integral of $F$ over a path from $x_1$ to $x_2$ is independent of the piecewise smooth curve $\gamma$ from $x_1$ to $x_2$, and so can be written
\[
\int_{x_1}^{x_2} F \cdot dx
\]
\item For every piecewise smooth close curve lying in $D$
\[
\oint F\cdot dx = 0
\]
\end{enumerate}
\end{theorem}

\begin{theorem}
Let $R$ be an open coordinate rectangle in $\R^n$, and let $F$ be a continuously differentiable vector field on $R$. If $F'(x)$, the Jacobian matrix of $F$, is symmetric on $R$, then $F$ is a gradient field.
\end{theorem}

\begin{theorem}[Flux]
\[
\Phi = \int_S F \cdot dS = \int_D F \cdot d\omega
\]
\end{theorem}

\begin{definition}
$g(u,v)$ is a parametrization of surface $S$.
Area
\[
\sigma(S) = \int_D |\frac{\partial g}{\partial u}(u,v) \times \frac{\partial g}{\partial v}(u,v)|du dv
\]
area Integral
\[
\int_S p d\sigma = \int_D p(g(u,v))|\frac{\partial g}{\partial u}(u,v) \times \frac{\partial g}{\partial v}(u,v)|du dv
\]
\end{definition}

\begin{definition}
Surface integral of $F$ over $S$. Coordinate $g(u,v)$
\[
\int_S F\cdot dS
\]
\[
=\int_D F(g(u,v))\cdot (\frac{\partial g}{\partial u}(u,v) \times \frac{\partial g}{\partial v}(u,v))du dv
\]
\[
=\int_D F_1 \frac{\partial(x_2,x_3)}{\partial(u,v)} +  F_2 \frac{\partial(x_3,x_1)}{\partial(u,v)}+ F_3 \frac{\partial(x_1,x_2)}{\partial(u,v)}
\]
\[
=\int_D F_1 dx_2 dx_3 + F_2 dx_3dx_1+F_3 dx_1 dx_2
\]
\end{definition}

\begin{theorem}
\[
\int_S \curl F\cdot dS = \oint_{\partial S} F\cdot dx
\]
\end{theorem}

\begin{definition}
\[
\curl F(x) = \left(\frac{\partial F_3}{\partial x_2}(x)-\frac{\partial F_2}{\partial x_3}(x),
\frac{\partial F_1}{\partial x_3}(x)-\frac{\partial F_3}{\partial x_1}(x),
\frac{\partial F_2}{\partial x_1}(x)-\frac{\partial F_1}{\partial x_2}(x)\right)
\]
\end{definition}

\begin{theorem}
\[
\int_R \operatorname{div} F dV = \int_{\partial R} F\cdot dS
\]
\[
\int_R (\frac{\partial F_1}{\partial x_1}+ \frac{\partial F_2}{\partial x_2}+ \frac{\partial F_3}{\partial x_3}) dx_1dx_2dx_3
= \int_{\partial R} F_1 dx_2 dx_3 + F_2 dx_3 dx_1 + F_3 dx_1 dx_2
\]
\end{theorem}

\begin{theorem}[General Stokes' theorem]
\[
\int_M d\omega = \oint_{\partial M} \omega
\]
\end{theorem}

\begin{theorem}
Work form, or circulation form is
\[
\omega = F_1 dx + F_2 dy + F_3 dz
\]
Flux form is
\[
\omega = F_1 dy\wedge dz - F_2 dx\wedge dz + F_3 dx\wedge dy
\]
Volume form for surface in coordinate $u,v$ is
\[
\sqrt{g} du\wedge dv
\]
where $g = det (g_{ij})$.\\
Density form for a density $f(u,v)$ is $f$ times the volume form.
\end{theorem}


\end{document}
\theend
