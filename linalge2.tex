\documentclass[letter]{article}
\usepackage{amsmath}
\usepackage{amssymb}
\usepackage{amsfonts}
\usepackage{amsthm}
\usepackage{latexsym}
\usepackage{mathrsfs}
\usepackage{eufrak}
\usepackage[pdftex]{graphicx}
\usepackage{color}

\setlength{\topmargin}{-0.5in} \setlength{\textwidth}{6.5in}
\setlength{\oddsidemargin}{0.0in} \setlength{\textheight}{9.1in}

\newlength{\pagewidth}
\setlength{\pagewidth}{6.5in} \pagestyle{empty}

\newcommand{\R}{\mathbb{R}}
\newcommand{\N}{\mathbb{N}}
\newcommand{\B}{\mathfrak{B}}
\newcommand{\Z}{\mathbb{Z}}
\newcommand{\im}{\operatorname{im}}
\newcommand{\image}{\operatorname{image}}
\newcommand{\rref}{\operatorname{rref}}
\newcommand{\rank}{\operatorname{rank}}
\newcommand{\nullity}{\operatorname{nullity}}
\newcommand{\proj}{\operatorname{proj}}

%\newcommand{\ker}{\operatorname{ker}}
\newtheorem{theorem}{Theorem}[section]
\newtheorem{lemma}[theorem]{Lemma}
\newtheorem{proposition}[theorem]{Proposition}
\newtheorem{corollary}[theorem]{Corollary}

\newenvironment{definition}[1][Definition]{\begin{trivlist}
\item[\hskip \labelsep {\bfseries #1}]}{\end{trivlist}}
\newenvironment{example}[1][Example]{\begin{trivlist}
\item[\hskip \labelsep {\bfseries #1}]}{\end{trivlist}}
\newenvironment{remark}[1][Remark]{\begin{trivlist}
\item[\hskip \labelsep {\bfseries #1}]}{\end{trivlist}}

\title{Linear Algebra(Bretscher) Chapter 2 Notes}
\date{}

\begin{document}
\maketitle
\vspace{-.5in}
\section{List of linear transformations}
\subsection{Scalings}
$\begin{bmatrix}
k&0\\
0&k
\end{bmatrix}$ is a scaling of a factor of $k$.

\subsection{Orthogonal Projections}
Any vector $\vec x$ in $\R^2$ can be uniquely written as
\[
\vec x = \vec x^\parallel + \vec x^\perp
\]
where $\vec x^\parallel$ and $\vec x^\perp$ are parallel and perpendicular to $L$, respectively.\\
The transformation $T(\vec x) =  \vec x^\parallel$ is the orthogonal projection of $\vec x$ onto $L$.
$\vec w$ is a nonzero vector parallel to $L$.$\vec u$ is a unit vector parallel to $L$.
\[\proj_L(\vec x) = \vec x^\parallel =  \frac{\vec x \cdot \vec w}{\vec w \cdot \vec w} \vec w = (\vec x \cdot \vec u) \vec u\]
The matrix for $\proj_L(\vec x)$ is

\[\frac{1}{w_1^2+w_2^2}
\begin{bmatrix}
w_1^2&w_1w_2\\
w_1w_2&w_2^2
\end{bmatrix} = \begin{bmatrix}
u_1^2&u_1u_2\\
u_1u_2&u_2^2
\end{bmatrix}
\]

Let $L^\perp$ be the line perpendicular to $L$, then \[\proj_{L^\perp}(\vec x) = \vec x^\perp \]
\subsection{Reflections}


\[\operatorname{ref}_L(\vec x) \]
\[=  \vec x^\parallel - \vec x^\perp\]
\[= \vec x - 2\vec x^\perp\]
\[= 2 \vec x^\parallel - \vec x\]
\[= 2 \proj_L(\vec x) - \vec x\]
\[= 2(\vec x \cdot \vec u)\vec u - \vec x\]

The matrix is in the form 
$ \begin{bmatrix}
a&b\\
b&-a
\end{bmatrix}$, where $a^2+b^2 = 1$

\subsection{Rotations}
The matrix of a counterclockwise rotation in $\R^2$ though an angle $\theta$ is
\[
\begin{bmatrix}
\cos \theta&-\sin \theta\\
\sin \theta&\cos \theta
\end{bmatrix} 
= 
\begin{bmatrix}
a&-b\\
b&a
\end{bmatrix} 
\]
where $a^2+b^2 = 1$.

\subsection{Rotations combined with scaling}
$\begin{bmatrix}
a&-b\\
b&a
\end{bmatrix}$ is a rotation of $\theta$ and scale of $r$, where $a = r\cos\theta$, $b=r\sin\theta$.

\subsection{Shears}
A horizontal shear is the form $\begin{bmatrix}
1&k\\
0&1
\end{bmatrix}$, a vertical shear is the form $\begin{bmatrix}
1&0\\
k&1
\end{bmatrix}$

\section{The inverse of a linear transformation}
Inverse of a $2\times 2$ matrix 
\[
\begin{bmatrix}
a&b\\
c&d
\end{bmatrix}^{-1} = \frac{1}{ad-bc}
\begin{bmatrix}
d&-b\\
-c&a
\end{bmatrix}
\]

\[(BA)^{-1} =
A^{-1}B^{-1}
\]

\end{document}
\theend
